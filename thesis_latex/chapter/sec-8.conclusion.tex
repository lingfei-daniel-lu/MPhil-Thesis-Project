\chapter{Conclusion}\label{sec-conclusion}

In this paper, we provide evidence at a highly disaggregated level for the incomplete import exchange rate pass-through in China. Our research contributes to the literature by revealing how importers' characteristics, especially the degree of financial constraints that they face, affect exchange rate price pass-through patterns. Utilizing unit value information from Chinese Customs and comparing imports with exports, we find that (1) the average import price pass-through in China is around 35-40\%, far below the 95\% export price pass-through; (2) for firms in industries with more stringent credit constraints, both import and export exchange rate pass-through tend to be more complete; (3) import source diversity can effectively reduce import price pass-through and offset the effects of credit constraints. The main novelty of our empirical strategy is to focus on the role of importers. We believe that micro import price pass-through measures China's ability to withstand risks in the international trade market from a new perspective.

There are several directions for future improvement. First, we need to explore the underlying mechanism by which credit constraints affect exchange rate pass-through. We only verify this effect based on a reduced-form approach at this stage. Even after controlling for some potential channels claimed by literature, we are not yet clear about how the remaining effects of credit constraints work. Future work should build a structural model to identify the detailed channels. Second, we could study how a firm's import and export behaviors influence each other. The dominance of two-way traders in China's international trade volume is a key fact that we cannot ignore. Adjustments on the import side and export side are two sides of the same coin for companies to face exchange rate shocks. Third, we should pay attention to the trend of China's exchange rate pass-through over time. The trend may reflect changing market power of Chinese firms and their patterns of pricing to market behaviors. Ideally, we expect to distinguish the contribution of each factor to the trend in exchange rate pass-through. 