\chapter{Empirical Framework}\label{sec-4.framework}
This section describes our econometric specifications and measurements of key variables of interest.

\section{Estimating Equations}

\subsection{Baseline estimations of exchange rate pass-through}

Since exchange rate pass-through is not an indicator that can be directly measured, we need to use panel data regression to estimate it. The first step goal is to estimate exchange rate pass-through as the elasticity of unit values changes to exchange rate changes using the firm-product-country details. Our strategy is based on the fixed effects regression. Specifically, we run a regression of import price changes on the bilateral real exchange rate changes between China and the source country, controlling the change of real GDP in the source country. The baseline equation refers to AIK\cite{aik2014} and LMX\cite{lmx2015} as below:

\begin{equation}
	\Delta \ln P^{D}_{i j c t}=\alpha+\beta^D \Delta \ln R E R_{c t}+\gamma \Delta \ln R G D P_{c t}+\xi_{i j c}+\tau_{t}+\varepsilon_{i j c t}
	\label{eq4.1}
\end{equation}

where $P_{ijct}$ represents the trade price of the product $i$ bought (sold) by firm $j$ from (to) country $c$ during year $t$.  $D \in$ \{Import, Export\} denotes the trade direction. Therefore, this equation could be used to estimate both import and export exchange rate pass-through with different values of D. $R E R_{c t}$ is the bilateral real exchange rate between Chinese RMB and currency in the country $c$. $RGDP_{ct}$ represents the real GDP of the source country deflated to the constant price level, which proxy for market demand. $\xi_{ijc}$ denotes the firm-product-country level fixed effects to capture any time-invariant unobserved factors for a combination of firm, product, and destination. These multi-dimensional fixed effects restrict unit value changes to price adjustments, rather than other changes in corporate trade decisions. $\tau_t$, the year dummies, control for macro-shocks that are common to all firms. We will alter the fixed effects setting in robustness checks.

To deal with possible non-stationarity of the panel, we use the first difference of the logarithms for prices $\Delta \ln P^{D}_{i j c t}$, real exchange rates $\Delta \ln R E R_{c t}$ and real GDP $\Delta \ln R G D P_{c t}$ to represent their annual rates of change. In this way, we transform the dynamic panel into a fixed effects regression. Therefore, using import price changes, the estimated coefficient of interest $\beta$ is the elasticity of price changes to exchange rate changes, i.e. import exchange rate pass-through. We also provide estimations for export exchange rate pass-through which is more common in past literature, where import prices are now replaced by export prices and the information of sources is replaced by which of destination markets. 

The real prices for export and import $P^{Import}_{i j c t}$ and $P^{Export}_{i j c t}$ are both denominated by the Chinese RMB in this paper. Using RMB as the denomination currency will make the coefficients on the import side and export side have different meanings. The level of coefficient $\beta^{D=Import}$ measures the completeness of import exchange rate pass-through, i.e. a higher $\beta$ means Chinese importers face more volatile import RMB prices during exchange rate shocks. However, for the export side,  $\beta^{D=Export}$ means the "incompleteness" of the export pass-through because a higher $\beta$ means Chinese exporters pass less exchange rate change to the destination market while having more volatile domestic currency prices. To be noticed, a majority of firms in our sample are both importers and exporters so they will appear in estimations of both exchange rate pass-through.

\subsection{Estimations with credit constraints}\label{seq-4.1.2}

Since our main focus is how firms' credit constraints affect exchange rate pass-through, we then include an interaction term of sectors’ financial vulnerability into the estimation function. Intuitively, firms operating in those financially vulnerable industries tend to have less access to enough funds to support their international trade activities, that is, they are subject to tighter credit constraints.

Therefore, we study the credit constraint effects on exchange rate pass-through across sectors with the following panel regression:

\begin{equation}
	\Delta \ln P^{D}_{ijct}=\alpha+\beta^D_{1} \Delta \ln RER_{ct}+\beta^D_{2} \Delta \ln RER_{ct} \cdot FV_{j}+\gamma \Delta \ln RGDP_{ct}+\xi_{ijc}+\tau_{t} +\varepsilon_{ijct}
	\label{eq4.2}
\end{equation}

where the variable $FV_{j}$ represents the financial vulnerability of the sector to which the firm $j$ belongs and the rest are the same as those in the baseline equation. The interaction coefficient $\beta_2$ represents the effect of credit constraints on exchange rate pass-through. A positive $\beta^{Import}_2$ for importers implies more credit-constrained importers have a more complete import exchange rate pass-through, while a positive $\beta^{Export}_2$ implies more credit-constrained exporters have a less complete exchange rate pass-through. The overall import ERPT for an importer $j$ is given by $\beta^D_{1} +\beta^{Import}_{2} FV_j$ and the export ERPT for an exporter $j$ is  $\beta^D_{1} +\beta^{Export}_{2} FV_j$.

Through this estimation strategy, we hope to scrutinize how the pricing behavior of Chinese importers in response to the exchange rate is affected by credit constraints and compare it with that of exporters. Although the functional forms for export and import pass-through are similar, the underlying mechanism could be different, which is one of our key innovation points. While credit-constrained exporters’ pricing decisions to deal with exchange rate shocks are mainly related to production and profit margin, the penetration effect of credit constraints on import prices is through a more direct channel, as the shortage of funds directly affects purchasing choices and bargaining.

\subsection{Estimations with additional factors}\label{sec-4.1.3}

After estimating exchange rate pass-through at the firm level and the impact of credit constraints on it, we need to go a step further to explore why. Through what channels will credit constraints affect the ability of importers to cope with exchange rate shocks? What other factors would exacerbate or diminish this effect? Are the effects of credit constraints fully explained by these controls?

In this part, we explore additional factors that affect firm-level import exchange rate pass-through. To do so, we introduce a vector $\mathbb{Z}_{jt}$ (or its lagged form $\mathbb{Z}_{jt-1}$) to include those additional factors and apply it to both control terms and interaction terms with real exchange rate changes: 

\begin{equation}
	\Delta \ln P^{D}_{ijct}=\alpha+[\beta^D_{1}+ \beta^D_{2} \cdot FV_{j}+\beta^D_{3} \cdot {\mathbb{Z}^{D}_{jt-1}}'] \Delta \ln RER_{ct} +\gamma \Delta \ln RGDP_{ct}+ {\mathbb{Z}^{D}_{jt}}' \eta+\xi_{ijc}+\tau_{t} +\varepsilon_{ijct}.
	\label{eq4.3}
\end{equation}

\begin{equation}
	\begin{aligned}
		\Delta \ln P^{D}_{ijct}=&\alpha+[\beta^D_{1}+ \beta^D_{2} \cdot FV_{j}+\beta^D_{3} \cdot {\mathbb{Z}^{D}_{jt-1}}'+\beta^D_{4} \cdot FV_{j} \cdot {\mathbb{Z}^{D}_{jt-1}}'] \Delta \ln RER_{ct} \\ &+\gamma \Delta \ln RGDP_{ct}+ {\mathbb{Z}^{D}_{jt}}' \eta+\xi_{ijc}+\tau_{t} +\varepsilon_{ijct}.
	\end{aligned}	
	\label{eq4.4}
\end{equation}

We will then use the estimation strategy of the form \ref{eq4.3} and \ref{eq4.4} to take into account various factors that may directly or indirectly affect exchange rate pass-through. We want to verify whether the effects of credit constraints are fully explained by some of these channels. Our firm-level data has the merit of containing information on its production, so we could connect the exchange rate pass-through with estimated firm-specific markup. We will also include trading partners later to measure the flexibility to choose alternative import sources. The coefficient of the interaction term between additional factors and real exchange rate movement $\beta_3$ represents the direct effects of those factors on the exchange rate pass-through other than through financial constraints. In equation \ref{eq4.4}, the triple interaction coefficient represents the indirect effects of those factors on the exchange rate pass-through through financial constraints. The same sign of $\beta_2$ and $\beta_4$ means that the factor enhances the effect of credit constraints, while the opposite sign means that it alleviates credit constraints.

Market share is another popular proxies for firm size or market power. For example, AIK (2014)\cite{aik2014} uses destination-specific market shares proxying for markup elasticity. When using market share as the additional factor, equation \ref{eq4.3} becomes the derivative form as below:

\begin{equation}
	\begin{aligned}
	\Delta \ln P^{D}_{ijct}=&\alpha+[\beta^D_{1}+ \beta^D_{2} \cdot FV_{j}+\beta^D_{3} \cdot S^{D}_{ijct}+\beta^D_4 \cdot {S^{D}_{ijct}}^2] \Delta \ln RER_{ct} \\
	&+\gamma \Delta \ln RGDP_{ct}+ \eta S^{D}_{ijct}+\xi_{ijc}+\tau_{t} +\varepsilon_{ijct}.
	\end{aligned}
	\label{eq4.5}
\end{equation}

The quadratic term for market share is used to test whether there is a non-monotonic relationship between exchange rate pass-through and market shares, such as the U-shape relationship documented by \cite{garetto2016} and \cite{devereux2017}. In addition to this, controlling for market share improves our estimation accuracy of the effect of credit constraints on exchange rate pass-through.

\section{Measurements}

\subsection{Unit value as trade price}

The customs records contain disaggregated trade values (denominated by US dollars) and quantities for each HS6 product $i$, each firm $j$, from (or to) each country $c$, in each year $t$, $V_{ijct}$, and $Q_{ijct}$. We first convert the value of the goods into RMB using the average exchange rate for the year. Then, the import and export prices we use are computed as unit values, defined as 

$$
P^{D}_{ijct}=\frac{V^{D}_{ijct}\cdot NER_{US,t}}{Q^{D}_{ijct}}
$$

where $D \in$ \{Import, Export\} and $NER_{US,t}$ is the annualized nominal exchange rate of US dollars in terms of RMB in year $t$. Because product categories are highly subdivided, we believe that the unit value is an ideal proxy for the transaction price.

Similar to real exchange rates, we take the first difference of the logarithm to represent price changes of a certain product across years. We will exclude observations with the annual growth rate of unit value in the top or bottom 1 percentile in the distribution, by HS2 product category and year, to avoid results being affected by extreme idiosyncratic factors other than some exchange rate adjustments.

\subsection{Credit constraints}\label{sec-4.2.2}

One of the most critical issues in our empirical strategy is to measure the extent of financial constraints. To deal with potential concurrent endogeneity, our measures of credit constraints are applied to each firm across the whole period. Following a widely recognized literature on the role of credit constraints in international trade (\cite{kroszner2007}; \cite{manova-wei-zhang2015}; \cite{fan-lai-li2015}), we use multiple financial vulnerability measures at the sector level to proxy for credit needs (demand for outside capital) and ability to resist financial risks. These measures are designed to reflect the nature of each industry which should be regarded as exogenous for each firm. If a firm is in a more financially vulnerable industry, it tends to face a tighter credit constraint, regardless of its operating conditions.

The first measure we use is external finance dependence ($ExtFin_j$), the share of capital expenditures not financed by operational cash flows. If external finance dependence is high, the industry is more financially vulnerable and firms in this industry are more credit constrained. The second measure is asset tangibility ($Tang_j$), which describes the share of the net value of tangible assets that firms can pledge as collateral to raise external finance, in its total book value. The third measure is the inventory-to-sales ratio ($Invent_j$), which measures the production cycle duration and the necessary working capital to maintain inventories and meet demand.  

To utilize the U.S. industry-level credit measures in the literature, we match the CIC industry code system used in China to the International Standard Industrial Classification (ISIC) system. We first convert the older ISIC Revision 2 3-digit and 4-digit industries from \cite{manova-wei-zhang2015} to match the newest ISIC Revision 3 codes; then we link the ISIC Revision 3 codes to the adjusted CIC codes in CIE datasets. Finally, we could match firms in the merged sample to those sector-level financial vulnerability measures. One-to-many situations may occur in the process of encoding matching. In this case, we construct the target variable for the new industry by averaging its source industries.

Although we construct three measures of credit constraints as in the literature, we will focus on the external finance dependence and tangibility in our later analysis. One important reason is that their interpretation can be linked to firms' exposure and resistance to financial frictions directly. In contrast, the inventory ratio may be connected to inventory management efficiency rather than liquidity and financial reasons. Following \cite{manova-wei-zhang2015}, we also construct the first principal component of external finance dependence and asset tangibility $FPC_j$, which increases with the former and falls with the latter. An industry with a higher $FPC_j$ is more financially sensitive if firms in it require more outside funds but own less collateralizable assets. Therefore, we could use $FPC_j$ as an aggregate measure to combine information about financial vulnerability from $ExtFin_j$ and $Tang_j$.

We have two major reasons why we use credit constraint measures based on US data in our main regressions. First, we want to remove the distortion by the limited credit supply in China and focus on the credit demand associated with sectoral characteristics. Second, the U.S. patterns of sectoral credit demand are proved persistent in a cross-country setting in the literature (\cite{kroszner2007}; \cite{manova-wei-zhang2015}; \cite{fan-lai-li2015}), especially when the industry classification is broadly defined. Intuitively, the financial needs of an industry may differ in level across countries, but the relative ranking between industries is supposed to be the same across countries, due to technical reasons specific to the industry itself.

Alternatively, we also compute credit needs based on Chinese firm-level information from CIE data. In addition to the already mentioned three measures, external finance dependence ($ExtFin_j$), asset tangibility ($Tang_j$), and inventory ratio ($Invent_j$), we include the fourth measure is R\&D intensity ($RD_j$), defined as the ratio of research and development expenditure to the total sales. Usually, R\&D activities are capital-intensive so it requires firms to pay a large fixed cost before production and sales. Therefore, firms in an R\&D-intensive industry should be more financially vulnerable. However, since we only have the information on firms' R\&D expenditure in and after 2005, which narrows the range of available samples, we will only use R\&D intensity as an auxiliary proxy variable. 

We adopt the measure of external finance dependence used by \cite{fan-lai-li2015}. Then we calculate the inventory ratio as the value of inventory over sales income, the asset tangibility as the value of fixed assets over total assets, and the R\&D intensity as R\&D spending over total sales income. To avoid credit constraints being endogenously affected by other corporate factors, we take the median of the firm-level credit constraint measure in the same CIC 2-digit industry as the industry-level credit constraint measure. Since the R\&D investment of a considerable number of companies is equal to 0, we choose to take the average rather than the median when calculating the R\&D intensity of the industry. The regression results using the Chinese industry measures are provided as robustness checks.

The summary statistics of the credit constraints measures in the firm-level data are shown in panels A and B in Table \ref{tab4.1}, respectively.

\begin{table}[htbp]
	\centering
	\caption{Summary Statistics of Credit Constrains Measures}
	\label{tab4.1}
	\resizebox{1.05\textwidth}{32mm}{
		\begin{threeparttable}
			\begin{tabular}{lcccccc}
				\toprule
				& \multicolumn{1}{l}{\#observations} & \multicolumn{1}{l}{Mean} & \multicolumn{1}{l}{Median} & \multicolumn{1}{l}{Std. dev} & \multicolumn{1}{l}{P10} & \multicolumn{1}{l}{P90} \\
				\textbf{Panel A: US Measures} &       &       &       &       &       &  \\
				$FPC_{j}$ &  1,745,511 & -7.12e-09 & -0.2706642 & 1 & -1.071394   & 1.072687 \\
				$ExtFin_{j}$ & 1,745,511 & -.0036698 & -0.05 & 0.3112002 & -0.25 & 0.28 \\
				$Tang_{j}$ & 1,745,511 & 0.3106788 & 0.32 & 0.0944181 & 0.1866667 & 0.43 \\
				$Invent_{j}$ & 1,745,511 & 0.1594069 & 0.1633333  & 0.0292352 & 0.115   & 0.1933333 \\
				\midrule
				& \multicolumn{1}{l}{\#observations} & \multicolumn{1}{l}{Mean} & \multicolumn{1}{l}{Median} & \multicolumn{1}{l}{Std. dev} & \multicolumn{1}{l}{P10} & \multicolumn{1}{l}{P90} \\
				\textbf{Panel B: Chinese Measures} &       &       &       &       &       &  \\
				$ExtFin_{j}$ & 1,745,511 & -0.6479498 & -0.47 & 0.6746751 & -1.32 & -0.1  \\
				$Tang_{j}$ & 1,745,511 & 0.3332769 & 0.3268749 & 0.0648019 & 0.2390799 & 0.4317028 \\
				$Invent_{j}$ & 1,745,511 & 0.1102537 & 0.1030875  & 0.0274747 & 0.0778921   & 0.1348336 \\
				$R\&D_{j}$ & 1,745,511 & 0.0168278 & 0.012111 & 0.0142106 & 0.0053125  & 0.0281532 \\
				\bottomrule
			\end{tabular}
			\begin{tablenotes}
				\footnotesize
				\item[*] This table shows the summary statistics of credit constraint measures. Panel A describes the measures calculated using US data while panel B shows the alternative Chinese version. All variables are unitless, the numerical size only means relative rank.
			\end{tablenotes}
		\end{threeparttable}
	}
\end{table}

\subsection{Import sources and export markets}\label{sec-4.2.3}
Following the literature about import sourcing, an importer's sourcing diversity could increase its bargaining power in import prices in addition to its production characteristics. We want to test how importers' sourcing diversity affects exchange rate pass-through. We provide a simple measure for the firm-product level sourcing diversity $Source_{ijt}$ as the number of source countries from which an importer $j$ imports a certain HS6 product type $i$ in year $t$. 

Similarly, for exporters, we count the number of destination countries to which an exporter exports a certain HS6 product type $i$ as the firm-product-level selling measure, $Market_{ijt}$. Controlling other variables, the number of export markets for the same product can measure its export network diversity. In a robustness test, we use the firm-year fixed effect to control for differences in import and export diversity caused by firm size.

\subsection{Firm-level markup}\label{sec-4.2.4}

In the discussion, we argue that credit constraints will affect the "absorptive capacity" of exchange rate shocks other than the firm's attributes in sales. In the following work, we will control markup to test the conjectures concretely. Referring to \cite{bkl2021}, even without direct measures of prices and marginal cost, we can still estimate the firm-level markup, using the structural assumptions of \cite{dlw2012} (DLW hereafter) and GMM estimation method.

Simply put, DLW (2012)\cite{dlw2012} derives the firm-specific markup as the ratio of an input factor's output elasticity to its firm-specific factor payment share $\mu_{t}=\theta_{t}^{X}\left(\alpha_{t}^{X}\right)^{-1}$, where $\alpha_{t}^{X}$ is the share of expenditures on input X in total sales and $\theta^X_t$ denotes the output elasticity on an input X. The major difficulty is calculating the firm-specific output elasticity concerning materials, which requires estimating firm-specific production functions. We apply the methodology of \cite{acf2015} to address the endogeneity of inputs, assuming a 3rd-order translog gross output production function in capital, labor, and material inputs:

$$
\begin{aligned}
y_{t}= &\beta_{k} k_{t}+\beta_{l} l_{t}+\beta_{i} m_{t}+\beta_{k 2} k_{t}^{2}+\beta_{l 2} l_{t}^{2}+\beta_{m 2} m_{t}^{2}+\beta_{k l} k_{ t} l_{t}+\beta_{k m} k_{t} m_{t}+\beta_{l m} l_{t} m_{t}+\beta_{k 3} k_{t}^{3}\\
&+\cdots+\omega_{t}+\epsilon_{t}.
\end{aligned}
$$

In practice, we need to construct four production variables in log form: real output value $y_t$, persons engaged $l_t$, real fixed assets at current value $k_t$, and real material inputs $m_t$. Real output values are deflated by output deflators, while real fixed assets and real material inputs are deflated by investment deflators and input deflators, respectively. The deflators are constructed as in \cite{brandt2012}.

\subsection{Market share}\label{sec-4.2.5}

In addition to the extensive diversity measured by the number of import sources or export markets, we also use a firm's share in a specific import or export market to describe its intensive competitiveness.

Following AIK\cite{aik2014} and \cite{devereux2017}, we define import market share as a firm’s value share in the import market, within a given HS6 product category. Therefore, a single firm can have multiple import market shares for multiple products. Our definition of import market share is also year specific, and so a firm’s import market share can vary over time. 

$$
S^{D}_{ijct} \equiv \frac{v^{D}_{ijct}}{\sum_{j^{\prime} \in J_{ict}} v^{D}_{ij^{\prime}ct}}
$$

where $D \in$ \{Import, Export\}. The capital letter $J$ denotes the set of potential competitors in the same product-specific market. 

The destination-specific export market share proxy is similarly defined as the value share of a firm relative to all Chinese exporters in our sample who export the same product to the same market. Since we only have data from China Customs, our export market share $S^{D}_{ijct}$ is relative to other Chinese firms. The external competitive stance in a particular sector-destination pair is also common for all Chinese exporters in that country and hence our measure captures all relevant variation in market share between firms in our sample.