\begin{abstract}

This paper shows the incomplete exchange rate pass-through patterns and its linkage with importers' characteristics. Using Chinese firm-level information and customs records from 2000 to 2007, we find that (1) the average import price pass-through in China is about 35-40\%, far below the near complete 95\% export price pass-through; (2) for firms in financially more vulnerable industries, both import and export exchange rate pass-through tend to be more complete; (3) higher import source diversity from countries can effectively reduce import price pass-through and offset the effects of credit constraints, while higher productivity does not or even has the opposite effect. Our main innovation is to focus on the role of importers in the determination of international prices. We believe that micro-evidence from import exchange rate pass-through provides a new perspective to study the "exchange rate disconnect" puzzle.

\end{abstract}
