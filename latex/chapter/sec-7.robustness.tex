\chapter{Robustness}\label{sec-7.robustness}

\section{Alternative Measures of Credit Constraints}

As the first robustness test to verify our baseline results, we use alternative credit constraint measures from CIE database. The purpose is to avoid potential bias from differences in the attributes of industry credit demand in different countries. The details of constructing these Chinese variables are discussed in in section \ref{sec-4.2.2}. Although Chinese financial market is less mature than that of the US, the rankings of industries in credit constraints are comparable. Thus, the results based on the credit constraints measures from Chinese data are expected be consistent with our main findings. Our results are reported in table \ref{tab7.1} which can be easily compared with the results using US measures.

\begin{table}[htbp]
	\centering
	\caption{Alternative Estimations with Chinese Measures of Credit Constraints}
	\begin{threeparttable}
	\begin{tabular}{lcccc}
		\toprule
		& (1)   & (2)   & (3)   & (4) \\
		\midrule
		Panel A & \multicolumn{4}{c}{Import} \\
		& External Finance & Tangibility & Inventory & R\&D Intensity \\
		\midrule
		$\Delta \ln RER_{ct}$ & 0.502*** & 2.032*** & -0.752*** & 0.043** \\
		& (0.018) & (0.053) & (0.043) & (0.019) \\
		$\Delta \ln RGDP_{ct}$ & 0.249*** & 0.240*** & 0.333*** & 0.304*** \\
		& (0.090) & (0.090) & (0.090) & (0.090) \\
		$\Delta \ln RER_{ct}$*$ExtFin_{j}$ & 0.223*** &       &       &  \\
		& (0.014) &       &       &  \\
		$\Delta \ln RER_{ct}$*$Tang_{j}$ &       & -5.776*** &       &  \\
		&       & (0.176) &       &  \\
		$\Delta \ln RER_{ct}*Invent_{j}$ &       &       & 9.797*** &  \\
		&       &       & (0.352) &  \\
		$\Delta \ln RER_{ct}*R\&D_{j}$ &       &       &       & 16.398*** \\
		&       &       &       & (0.573) \\
		Year FE  & Yes   & Yes   & Yes   & Yes \\
		Firm-product-country FE & Yes   & Yes   & Yes   & Yes \\
		Observations & 1792020 & 1792020 & 1792020 & 1792020 \\
		\midrule
		Panel B & \multicolumn{4}{c}{Export} \\
		& External Finance & Tangibility & Inventory & R\&D Intensity \\
		\midrule
		$\Delta \ln RER_{ct}$ & 0.016** & -0.135*** & 0.086*** & 0.043*** \\
		& (0.007) & (0.023) & (0.019) & (0.007) \\
		$\Delta \ln RGDP_{ct}$ & -0.081** & -0.082** & -0.081** & -0.082** \\
		& (0.037) & (0.037) & (0.037) & (0.037) \\
		$\Delta \ln RER_{ct}$*$ExtFin_{j}$ & -0.021*** &       &       &  \\
		& (0.007) &       &       &  \\
		$\Delta \ln RER_{ct}$*$Tang_{j}$ &       & 0.557*** &       &  \\
		&       & (0.074) &       &  \\
		$\Delta \ln RER_{ct}*Invent_{j}$ &       &       & -0.504*** &  \\
		&       &       & (0.165) &  \\
		$\Delta \ln RER_{ct}*R\&D_{j}$ &       &       &       & -0.627*** \\
		&       &       &       & (0.243) \\
		Year FE  & Yes   & Yes   & Yes   & Yes \\
		Firm-product-country FE & Yes   & Yes   & Yes   & Yes \\
		Observations & 1793974 & 1793974 & 1793974 & 1793974 \\
		\bottomrule
	\end{tabular}
	\label{tab7.1}
	\begin{tablenotes}
		\footnotesize
		\item[*] Standard errors in parentheses; *, **, and *** indicate significance at 10\%, 5\% and 1\% levels. The dependent variable is the price change $\Delta \ln P_{ijct}$. Columns (1)-(4) use different measures of credit constraints calculated using Chinese data. All regressions include firm-product-country fixed effects and year fixed effects.
	\end{tablenotes}
	\end{threeparttable}
\end{table}

Columns 1-4 of panel A of table \ref{tab7.1} present the import-side results for external finance dependence, tangibility, inventory ratio, and R\&D intensity, respectively. Panel B report the export-side results with the same variables. All regressions include firm-product-country fixed effects and year fixed effects as before. Nevertheless, most of the interaction term coefficients exhibit the same signs as above, confirming the validity of our baseline findings of the effects of credit constraints on exchange rate pass-through. We can still conclude that financially more constrained firms (both importers and exporters) have more complete exchange rate pass-through than those less constrained, even with Chinese measures.

\section{Alternative Subsample: Two-way traders}

Since most of China's imports go through two-way traders, who conduct both export and import simultaneously, import exchange rate pass-through is likely to be related to export behaviors. Specifically, importers who also export may pass part of the price fluctuations of imported intermediate goods caused by exchange rate shocks to their export destination to buffer the impact of exchange rate risks. 

To test whether the pattern of China's import pass-through is mainly dominated by two-way traders, we use the sub-sample consisting only of two-way traders for a robustness check. Table \ref{tab7.2} presents the results for this restricted sample.

\begin{table}[htbp]
	\centering
	\caption{Alternative Estimations with Two-way Traders}
	\begin{threeparttable}
	\begin{tabular}{lcccc}
		\toprule
		& (1)   & (2)   & (3)   & (4) \\
		\midrule
		Panel A & \multicolumn{4}{c}{Import (Two-way traders)} \\
		& Baseline & FPC   & External Finance & Tangibility \\
		\midrule
		$\Delta \ln RER_{ct}$ & 0.394*** & 0.136*** & 0.231*** & 1.158*** \\
		& (0.015) & (0.016) & (0.015) & (0.031) \\
		$\Delta \ln RGDP_{ct}$ & 0.406*** & 0.459*** & 0.469*** & 0.427*** \\
		& (0.086) & (0.086) & (0.086) & (0.086) \\
		$\Delta \ln RER_{ct}$*$FPC_{j}$ &       & 0.388*** &       &  \\
		&       & (0.009) &       &  \\
		$\Delta \ln RER_{ct}$*$ExtFin_{j}$ &       &       & 1.246*** &  \\
		&       &       & (0.028) &  \\
		$\Delta \ln RER_{ct}$*$Tang_{j}$ &       &       &       & -3.138*** \\
		&       &       &       & (0.112) \\
		Year FE  &       & Yes   & Yes   & Yes \\
		Firm-product-country FE &       & Yes   & Yes   & Yes \\
		Observations & 1712289 & 1712289 & 1712289 & 1712289 \\
		\midrule
		Panel B & \multicolumn{4}{c}{Export (Two-way traders)} \\
		& Baseline & FPC   & External Finance & Tangibility \\
		\midrule
		$\Delta \ln RER_{ct}$ & 0.040*** & 0.051*** & 0.044*** & -0.034** \\
		& (0.006) & (0.006) & (0.006) & (0.016) \\
		$\Delta \ln RGDP_{ct}$ & -0.144*** & -0.145*** & -0.144*** & -0.147*** \\
		& (0.041) & (0.041) & (0.041) & (0.041) \\
		$\Delta \ln RER_{ct}$*$FPC_{j}$ &       & -0.022*** &       &  \\
		&       & (0.005) &       &  \\
		$\Delta \ln RER_{ct}$*$ExtFin_{j}$ &       &       & -0.048*** &  \\
		&       &       & (0.015) &  \\
		$\Delta \ln RER_{ct}$*$Tang_{j}$ &       &       &       & 0.284*** \\
		&       &       &       & (0.059) \\
		Year FE  &       & Yes   & Yes   & Yes \\
		Firm-product-country FE &       & Yes   & Yes   & Yes \\
		Observations & 1415415 & 1415415 & 1415415 & 1415415 \\
		\bottomrule
	\end{tabular}
	\label{tab7.2}
	\begin{tablenotes}
		\footnotesize
		\item[*] Standard errors in parentheses; *, **, and *** indicate significance at 10\%, 5\% and 1\% levels. The dependent variable is the price change $\Delta \ln P_{ijct}$. Columns (2)-(4) use different measures of credit constraints calculated using U.S. data. Panel A shows the estimation for import ERPT and effects of credit constraints on it while panel B shows those results for export ERPT. All regressions include firm-product-country fixed effects and year fixed effects.
	\end{tablenotes}
	\end{threeparttable}
\end{table}

Table \ref{tab7.2} shows that the results for the subset of two-way traders are highly similar to the results for the entire matched sample, indicating that the exchange rate pass-through pattern of two-way traders dominates China’s imports and exports. The results of using the two-way traders sample are still typical, whether estimating the exchange rate pass-through of imports and exports or examining the role of credit constraints. Combined with our discussion in section \ref{sec-6.3}, the nature of two-way traders should be the focus of future research on import-export connections.