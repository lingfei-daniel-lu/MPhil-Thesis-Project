\chapter{Introduction}\label{sec-1.introduction}

Why do exchange rate fluctuations not result in price changes of the same magnitude? This is one of the core questions among a set of "exchange rate disconnect" puzzles (\cite{obstfeld2000}). As price signals in the international trade market, exchange rates appear less informative for firms than expected. Exchange rate pass-through (ERPT), which describes the elasticity of local price changes to exchange rate fluctuations, varies widely across countries, industries, and time. Existing studies have generated widely varying estimates of the exchange rate pass-through. It is of particular interest to researchers both in the fields of international trade and open macroeconomics to demystifying the "exchange rate disconnect" puzzles. Understanding the pattern of exchange rate pass-through has important implications for formulating macro policy, including monetary policy, inflation targeting, and the balance of payments.

A large body of theoretical and empirical literature explores the mechanism of incomplete exchange rate pass-through in prices. Most micro explanations and empirical evidence based on disaggregated data for incomplete exchange rate pass-through focus on the exporter side. Exporters' productivity (\cite{bmm2012}; \cite{lmx2015}) and product quality (\cite{chen2016};\cite{auer2018}), as well as their imported inputs (\cite{aik2014}; \cite{wang-yu2021}) and market shares (\cite{auer2016}; \cite{devereux2017}), will all affect the export exchange rate pass-through. Yet the direct role of importers in determining exchange rate pass-through remains a novel field to study. 

Financial constraints, discussed in another strand of literature, also demonstrate influence on firms’ response in price-setting decisions to exchange rate fluctuations. \cite{strasser2013} first finds that financially constrained exporters adjust prices in the destination market more sharply when facing exchange rate shocks, implying a more complete exchange rate pass-through. Firms with tighter financial constraints tend to set higher prices due to higher external financing premiums (and the resulting higher marginal costs) and then face higher price elasticity of demand. Thus, with endogenously determined markups, exchange rate depreciation (appreciation) allows firms to increase (decrease) markups, but credit-constrained firms can do so only to a limited extent because they have less space to adjust their profit margins. According to common sense, the characteristics of buyers and sellers will both affect the sharing of international price risk. However, it remains an open question whether importers under financial constraints will behave differently in price negotiation during exchange rate shocks. Therefore, we want to introduce industry-level credit constraints into the study of trade price elasticity with exchange rate shocks.

In this paper, we focus on the role of importers in the determination of exchange rate pass-through and connect it with the importers' financial constraints. This paper tends to fill a gap in the literature by linking both sides of the trade relationship and provides a novel perspective to study the nature of exchange rate disconnect for emerging markets, where firms are more vulnerable to credit constraints due to immature financial markets. In contrast to the conventional framework of exchange rate pass-through where importers are mostly price takers, we contribute to the literature by identifying the importers' implicit sourcing power and comparing their heterogeneous capacity to absorb exchange rate shocks. Throughout the paper, we will compare firm-level import exchange rate pass-through with the export pass-through to reflect the similarities and differences between the two. Going a step further, if a country's export and import exchange rate pass-through patterns differ significantly, this may even affect terms of trade as well as current account imbalances. We believe that micro-evidence from import exchange rate pass-through provides a new perspective to study the "exchange rate disconnect" puzzle.

We estimate the exchange rate pass-through as the price elasticity of import prices concerning real exchange rates using Chinese micro-level data. Specifically, we merge the Chinese Industrial Enterprises datasets with the transaction data from China’s customs and adopted fixed effects panel regressions with first-order differences to capture the changes in product prices and real exchange rates. The average import exchange rate pass-through is between 35\%-40\%, which is obviously less complete compared to the over 95\% export pass-through. Second, we identify the effects of credit constraints on importers' exchange rate pass-through. We use US measures of sectors’ financial vulnerability (\cite{manova-wei-zhang2015}) in our main empirical analysis while use Chinese measures of credit needs (\cite{fan-li-yeaple2015}) for robustness checks. In our baseline results, import prices for firms in sectors with higher financial constraints are more sensitive to exchange rate shocks. Third, we calculate the proxy for an importer's sourcing base and find that an importer with more alternative importing options can resist the effects of credit constraints and absorb exchange rate shocks better.

In the discussion part, we control several potential factors which may affect the import pass-through other than credit constraints. We first estimate the firm-level markup estimated following \cite{dlw2012} and test how markups affect import ERPT. In addition, firms with different market shares also demonstrate heterogeneous price responses to exchange rate changes. We confirm that although firm heterogeneity in those aspects does affect exchange rate pass-through, credit constraints still play a role that cannot be ignored even after we control those firm characteristics. In robustness checks, we further use alternative measures of credit constraints and alternative subsamples. For alternative measures, we compare sector-level credit constraints variables calculated from China data with those from US data. For alternative samples, we divide our subjects into two subsets: two-way traders (simultaneous import and export) and one-way traders (either only import or only export). We check our results using only the two-way traders who account for the vast majority of trading firms and the vast majority of China's trade volume. These results are all significant and robust.

We show that importers‘ credit constraints do influence price-setting patterns in international trade. We provide evidence for three key findings: (1) the average import exchange rate pass-through level in China is significantly less complete than the export one; (2) financial constraints will increase both of them to be more complete, and (3) importers who import a certain product from more sources have a less complete pass-through. In other words, financially constrained importers will absorb more price fluctuations caused by exchange rate changes, while financially constrained exporters pass through more exchange rate changes to prices, both compared to those unconstrained firms. This reflects that binding financial constraints will lead to not only narrow margins to adopt pricing-to-market strategies for the sellers but also limited sourcing power for the buyers. Importers with a wider sourcing base could get access to alternative options and avoid bargaining disadvantages in exchange rate fluctuations to some extent.

The remainder of the paper is organized as follows. Section 2 presents a more detailed literature review. Section 3 describes the data we used. Section 4 introduces our empirical strategy and measures of key variables. Section 5 shows the main empirical results about import exchange rate pass-through and credit constraints. Section 6 discusses other factors affecting import exchange rate pass-through and directions for future research. Section 7 examines the robustness of the results. Section 7 concludes.

\newpage