\chapter{Literature Review} \label{sec-2.literature}

\section{Incomplete Exchange Rate Pass-through}

First, this paper contributes to a wide literature on exchange rate disconnect (\cite{obstfeld2000}),  particularly on the incomplete pass-through of exchange rate fluctuations into international prices. \cite{campa2005} document the lack of price sensitivity to exchange rate movements and provide estimates of the pass-through of exchange rates into import prices of 23 OECD countries. \cite{bmm2012} (henceforth, BMM) provide micro-level evidence of firm heterogeneity in response to real exchange rate shocks. Another milestone paper by \cite{aik2014} (henceforth, AIK) finds that firms with higher import intensity and larger market share have less complete exchange rate pass-through. Since then, more studies link the exchange rate price elasticity or pass-through to disaggregated firm-level characteristics.

We summarize three major channels leading to incomplete export price pass-through proposed by past literature. The first channel is local currency pricing (LCP). As surveyed by \cite{engel2002}, it means short-run nominal rigidities with prices sticky in the destination currency. Under LCP, firms that do not adjust prices have zero short-run pass-through. \cite{gopinath2008} introduce the stickiness and currency of pricing of traded goods into the discussion and provide direct evidence on the extent of LCP in US export and import prices from 1994 to 2005.  However, price rigidities can not fully explain the incomplete pass-through as shown in previous empirical results.

The second channel is pricing-to-market (PTM). It is derived from variable markups in which firms optimally set different prices depending on destination market conditions. \cite{atkeson2008} provide a quantitative investigation of the PTM channel and its implication for aggregate prices. \cite{bmm2012} finds that more productive firms respond to exchange rate changes by changing more markup rather than volume, which allows them to pass only a fraction of exchange rate shocks to their terminal prices. \cite{manova-zhang2012} also support that variable mark-ups and product quality across destinations should be considered to explain ERPT patterns. \cite{gopinath2010-currency} about currency choice and \cite{gopinath2010-frequency} about price adjustment frequency, further combine the above two channels in the discussion of ERPT. They show that the PTM and LCP channels of incomplete pass-through interact and reinforce each other. Firms subject to highly variable-markup endogenously choose to price in local currency as well as adopt longer price duration. If more firms do adjust prices infrequently, the exchange rate pass-through would decline.

The third channel is the marginal production cost.  While local distribution costs could result in incomplete pass-through into consumer prices, the imported inputs channel by \cite{aik2014} can affect producer factory gate prices. For example, an appreciation of home currency will decrease marginal costs due to cheaper imported inputs, thus partially offsetting the increases in export prices. The linkage between import behaviors and export pass-through helps us understand incomplete aggregate exchange rate pass-through and variation in pass-through across exporters. In addition, \cite{chatterjee2013} study the effect of exchange rate shocks on multi-product firms. They find that the extent of markup adjustments in response to exchange shocks would decline with the firm-product-specific marginal costs. This phenomenon implies the association of the second and third channels mentioned above.

For Chinese exchange rate pass-through, \cite{lmx2015} (LMX hereafter) find nearly complete exchange rate pass-through into RMB price for Chinese exporters, which means that their domestic currency price response is very weak. In other words, the overall degree of pricing-to-market is very low in China. According to the simplest price decomposition, $\Delta ln p = \Delta ln(MC) + \Delta ln(Markup)$, price adjustments can be broken down into marginal cost changes and markup adjustments. One popular explanation is that Chinese exporters have very limited profit margins to adjust their markup. However, this is inconsistent with the incomplete pass-through found in \cite{gopinath2008} using the US at-the-dock import prices. Both Chinese yuan export prices and U.S. onshore import prices vary very little with exchange rate fluctuations, possibly implying that Sino-U.S. trade does not dominate the overall exchange rate pass-through in the U.S. market. 

In addition, the complete ERPT to destination currency price is related to the “exorbitant privilege” of dominant currency in \cite{gopinath-stein2021}. Due to the dominance of the US dollar in international trade settlement, China's exports in the early time after WTO accession are mainly denominated in US dollars. A firm’s invoice currency choice, in turn, has a direct causal effect on the exchange rate pass-through into prices as in \cite{aik2022}. We observe little exchange rate fluctuations between US and China in the pegged period before 2005. At this time, there is no difference between Chinese exporters' pricing in RMB and US dollars, so exchange rate fluctuations between the destination currency and the US dollar are fully transmitted to the end market. Later, when China’s export market power had not yet been established, the pegged exchange rate was changed to a free-floating exchange rate. Exporters had to set prices in RMB more intensively, at the cost of temporarily abandoning the pricing-to-market strategy.

Recent literature reveals more firm-level evidence on heterogeneous pass-through. \cite{chen2016} predict more pricing-to-market and a smaller response of export volumes for higher quality goods and provide evidence with expert wine ratings to measure quality. \cite{garetto2016} raises two more arguments: 1) firm-level ERPT is a U-shaped function of firm-level productivity and market share; and 2) producers under incomplete information, such as new entrants, have less complete pass-through rates than those under complete information. Likewise, \cite{auer2016} also find a U-shaped relationship between the response of import prices to exchange rate changes and exporter market share with micro-data. \cite{devereux2017}'s novel feature is that ERPT and currency invoicing depend on the market share of both importers (negative) and exporters (U-shaped). It means exporters of extreme sizes (very small or very large) have higher pass-through rates and tend to invoice in foreign currency. Inspired by this, we will apply the idea of studying how exporters affect exchange rate pass-through to the study of importers. The characteristics of the importer are also the key factors affecting the structure and pricing characteristics of the international trade market.

\section{Credit Constraints and Trade}

Another important strand of literature discusses the effects of firms' credit constraints on international trade. This belongs to a broader field linking financial shocks and real economic activities. It is widely believed that exporters rely on extra external capital to pay the entry costs into foreign markets which can not be covered by internal cash flows from operations. There are two reasons for the additional external financial need to participate in trade: the act of exporting itself is riskier than domestic sales, and the fact that contractual reliability in international transactions is weaker (\cite{chaney2016}). Therefore, credit constraints for firms in financially vulnerable sectors and immature financial markets will largely be responsible for the "missing trade".

\cite{kroszner2007} classify firms by industries with varying degrees of external financial dependence when examining the impact of banking crises. \cite{manova2013} argues that credit constraints caused by financial market imperfections affect trade because only those firms that have sufficient liquidity to finance the additional expenditures for accessing foreign markets are able to export. \cite{chaney2016} defines financially constrained firms as those that lack both sufficient pledgeable assets and sufficient productivity to generate sufficient liquidity on their own. \cite{feenstra-li-yu2014}, \cite{manova-wei-zhang2015}, and \cite{fan-lai-li2015} provide comprehensive theoretical explanations and micro evidence from China about how credit constraints affect exports, through incomplete information, multinational links, and quality, respectively.

More recently, \cite{li-lan-ouyang2020} find that credit-constrained exporters respond to home currency depreciation by increasing production starting from sectors with lower external financing dependence until their limited financial resources are exhausted. Large revaluations during exchange rate fluctuations will also change their capacity for pledging collateral because of the unstable relative value of domestic and foreign assets (\cite{kohn2020}). 

As for the specific discussion of the relationship between credit constraints and exchange rate pass-through, \cite{strasser2013} uses a firm-level survey to show that financially-constrained firms tend to pass exchange rate shocks to prices to a more complete extent. He argues that borrowing constraints force firms to keep pricing-to-market (PTM) to a minimum as they do not have enough margin to adjust their markups, while unconstrained firms absorb more price shocks intentionally to maintain their optimal pricing policy. 

This article will improve two recent articles that use evidence from Chinese firms to discuss credit constraints and exchange rate pass-through (\cite{dai2021} and \cite{xu-guo2021}). Both studies verify that more financially constrained firms' exporting activities are more sensitive to exchange rate changes than those of less constrained firms, which is similar to the conclusion of \cite{strasser2013}. \cite{dai2021}'s analysis of the effect of access to finance on exports mostly follows the PTM channel while they focus more on aggregate export behaviors rather than the bilateral elasticity of export to each country. In response to the variable markup channel, \cite{xu-guo2021} further show that the effect of financial constraints on export value remains robust and significant besides the markup adjustment. 

\newpage