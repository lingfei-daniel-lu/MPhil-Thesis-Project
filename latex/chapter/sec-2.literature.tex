\chapter{Literature Review} \label{sec-2.literature}

\section{Incomplete Exchange rate pass-through}

First, this paper contributes to a wide literature on the exchange rate disconnect (Obstfeld and Rogoff, 2001\cite{obstfeld2000}) and particularly on the incomplete pass-through of exchange rate fluctuations into international prices. Campa and Goldberg (2005)\cite{campa2005} document the lack of sensitivity of prices to exchange rate movements and provide estimates of the pass-through of exchange rates into import prices of 23 OECD countries. Berman, Martin, and Mayer (2012)\cite{bmm2012} (henceforth, BMM) provide micro-level evidence of firm heterogeneity in response to real exchange rate shocks. Another milestone paper by Amiti, Itskhoki, and Konings (2014)\cite{aik2014} (henceforth, AIK) finds that firms with higher import intensity and market shares have lower exchange rate pass-through. Since then, more studies link the exchange rate elasticity or pass-through to disaggregated firm-level characteristics.

We summarize three major channels leading to incomplete export price pass-through proposed by past literature. The first channel is local currency pricing (LCP). As surveyed by Engel (2002)\cite{engel2002}, it means short-run nominal rigidities with prices sticky in the destination currency. Under LCP, the firms that do not adjust prices have zero short-run pass-through. Gopinath and Rigobon (2008)\cite{gopinath2008} introduce the stickiness and currency of pricing of traded goods into the discussion and provide direct evidence on the extent of LCP in US import and export prices from 1994 to 2005.  However, price rigidities can not fully explain the incomplete pass-through as shown in previous empirical results.

The second channel is pricing-to-market (PTM). It is from variable markups in which firms optimally set different prices depending on destination market conditions. Atkeson and Burstein (2008)\cite{atkeson2008} provide a recent quantitative investigation of the PTM channel and its implication for aggregate prices. BMM (2012)\cite{bmm2012} finds that more productive firms respond to exchange rate changes by changing more markup rather than volume, which allows them to pass only a fraction of exchange rate shocks to their terminal prices. Manova and Zhang (2012)\cite{manova-zhang2012} also support that variable mark-ups and product quality across destinations should be considered to explain ERPT patterns. Gopinath, Itskhoki, and Rigobon (2010)\cite{gopinath2010-currency} about currency choice and Gopinath and Itskhoki (2010)\cite{gopinath2010-frequency} about price adjustment frequency, further combine the above two channels. They show that the PTM and LCP channels of incomplete pass-through interact and reinforce each other, with highly variable-markup firms endogenously choosing to price in local currency as well as adopting longer price durations. If more firms were to adjust prices infrequently, the exchange rate pass-through would decline.

The third channel is the marginal production cost.  While local distribution costs could result in incomplete pass-through into consumer prices, the imported inputs channel by AIK (2014)\cite{aik2014} can affect producer factory gate prices. For example, an appreciation of home currency will decrease marginal costs due to cheaper imported inputs, thus partially offsetting the increases in export prices. The linkage between import behaviors and export pass-through helps understand low aggregate exchange rate pass-through and variation in pass-through across exporters. In addition, Chatterjee, Dix-Carneiro, and Vichyanond (2013)\cite{chatterjee2013} study the effect of exchange rate shocks on multi-product firms. They find that the extent of markup adjustments in response to exchange shocks would decline with the firm-product-specific marginal costs. This phenomenon implies the association of the second and third channels mentioned above.

Recent literature reveals more firm-level evidence on heterogeneous pass-through. Chen and Juvenal (2016)\cite{chen2016} predict more pricing-to-market and a smaller response of export volumes for higher quality goods and provide evidence with expert wine ratings to measure quality. Li, Ma, and Xu (2015)\cite{lmx2015} (LMX hereafter) find that the domestic price response of Chinese exporters to exchange rate changes is very weak, even though some more productive exporters price more to the market. Garetto (2016)\cite{garetto2016} raises two more arguments: 1) firm-level ERPT is a U-shaped function of firm-level productivity and market share; and 2) producers under incomplete information, such as new entrants, have lower pass-through rates than those under complete information. Likewise, Auer and Schoenle (2016)\cite{auer2016} also find a U-shaped relationship between the response of import prices to exchange rate changes and exporter market share with micro-data. Devereux, Dong, and Tomlin (2017)\cite{devereux2017}'s novel features are that ERPT and currency invoicing depend on the market share of both importers (negative) and exporters (U-shaped). It means very small or very large exporters have higher rates of pass-through and tend to invoice in the foreign currency.

\section{Credit constraints and trade}

Another important strand of literature discusses the effects of firms' credit constraints on international trade. This belongs to a broader field linking financial shocks and real economic activities. It is widely believed that exporters rely on extra external capital to pay the entry costs into foreign markets which can not be covered by internal cash flows from operations. Therefore, credit constraints for firms in financially vulnerable sectors and financially underdeveloped markets will largely be responsible for the "missing trade".

Kroszner, Laeven, and Klingebiel (2007)\cite{kroszner2007} classify firms by industries with varying degrees of external financial dependence when examining the impact of banking crises. Manova (2013)\cite{manova2013} and Chaney (2016)\cite{chaney2016} argue that credit constraints caused by financial market imperfections affect trade because only those firms that have sufficient liquidity to finance the additional expenditures for accessing foreign markets are able to export. Feenstra, Li, and Yu (2014)\cite{feenstra-li-yu2014}, Manova, Wei, and Zhang (2015)\cite{manova-wei-zhang2015}, and Fan, Lai, and Li (2015)\cite{fan-lai-li2015} provide theoretical explanations and micro evidence from China about how credit constraints affect exports, through incomplete information, multinational links, and quality, respectively. Large revaluations during exchange rate fluctuations will also change their capacity for pledging collateral because of the unstable relative value of domestic and foreign assets (Kohn, Leibovici, and Szkup, 2020\cite{kohn2020}). 

As for credit constraints and exchange rate pass-through, Strasser (2013)\cite{strasser2013} uses a firm-level survey to show that financially-constrained firms tend to pass through exchange rate shocks to prices to a more complete extent. He argues that borrowing constraints force firms to keep pricing-to-market (PTM) to a minimum as they do not have enough margin to adjust their markups, while unconstrained firms intentionally absorb more price shocks to maintain their optimal pricing policy. Li, Lan, and Ouyang (2020)\cite{li-lan-ouyang2015} find that credit-constrained exporters respond to home currency depreciation by increasing production starting from sectors with lower external financing dependence until their limited financial resources are exhausted.

This article will directly complement and improve on two recent articles that use evidence from Chinese firms to discuss credit constraints and exchange rate pass-through. Both Dai et al. (2021)\cite{dai2021} and Xu and Guo (2021)\cite{xu-guo2021} verify that more financially constrained firms' exporting activities are more sensitive to exchange rate changes than those of less constrained firms, which is similar to Strasser (2013)\cite{strasser2013}. Dai et al. (2021)\cite{dai2021}'s analysis of the effect of access to finance on exports mostly follows the PTM channel while they focus more on aggregate export behaviors rather than the bilateral elasticity of export to each country. In response to the variable markup channel, Xu and Guo (2021)\cite{xu-guo2021} further show that the effect of financial constraints on export value remains robust and significant besides the markup adjustment. 

\newpage
