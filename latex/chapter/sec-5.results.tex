\chapter{Empirical Results}\label{sec-5.results}
In this chapter, we will present our major empirical results using the samples and strategies described above. First, we show the estimated import exchange rate pass-through is rather incomplete while the export exchange rate pass-through is very close to complete. Second, we show that both importers and exporters with tighter credit constraints have more complete pass-throughs. Finally, we present the results with firm heterogeneity to explore factors affecting importers' capacity to absorb exchange rates and the role of credit constraints among them.

\section{Import Pass-through vs Export Pass-through}\label{sec-5.1}

Past literature on firm-level evidence of exchange rate pass-through mainly focuses on exporters' price-setting behaviors. Importers are likely to be more than simple price takers, which gives us a new perspective on exchange rate pass-through. Now we take a closer look at how real exchange rate fluctuations affect import prices and export prices differently in China. The results for import exchange rate pass-through versus export exchange rate pass-through are shown in Table \ref{tab5.1} using different samples.

\begin{table}[htbp]
	\centering
	\caption{Baseline Estimations of Exchange Rate Pass-Through}
	\begin{threeparttable}
	\begin{tabular}{lcccc}
		\toprule
		& (1)   & (2)   & (3)   & (4) \\
		\midrule
		Panel A & \multicolumn{4}{c}{Import} \\
		& Whole & Matched & Top 50 & Top 20 \\
		\midrule
		$\Delta \ln RER_{ct}$ & 0.179*** & 0.357*** & 0.354*** & 0.344*** \\
		& (0.003) & (0.015) & (0.015) & (0.016) \\
		$\Delta \ln RGDR_{ct}$ & -0.133*** & 0.263*** & 0.282*** & 0.333*** \\
		& (0.026) & (0.090) & (0.091) & (0.097) \\
		Year FE  & Yes   & Yes   & Yes   & Yes \\
		Firm-product-country FE & Yes   & Yes   & Yes   & Yes \\
		Observations & 8409682 & 1792020 & 1781948 & 1684798 \\
		\midrule
		Panel B & \multicolumn{4}{c}{Export} \\
		& Whole & Matched & Top 50 & Top 20 \\
		\midrule
		$\Delta \ln RER_{ct}$ & 0.050*** & 0.031*** & 0.039*** & 0.065*** \\
		& (0.002) & (0.005) & (0.006) & (0.009) \\
		$\Delta \ln RGDR_{ct}$ & -0.102*** & -0.083** & -0.118*** & -0.082 \\
		& (0.010) & (0.037) & (0.042) & (0.056) \\
		Year FE  & Yes   & Yes   & Yes   & Yes \\
		Firm-product-country FE & Yes   & Yes   & Yes   & Yes \\
		Observations & 11173463 & 1793974 & 1611410 & 1251147 \\
		\bottomrule
	\end{tabular}
	\begin{tablenotes}
		\footnotesize
		\item[*] Standard errors in parentheses; *, **, and *** indicate significance at 10\%, 5\% and 1\% levels. The dependent variable is the price change $\Delta \ln P_{ijct}$. Column (1) uses the long sample from 2000 to 2011. Column (2) uses the matched sample from 2000 to 2007. Columns (3) and (4) use finer sub-samples with only top 50 and top 20 partners ranked by total trade value. Panel A shows the results of import ERPT while panel B shows the results of export ERPT. All regressions include firm-product-country fixed effects and year fixed effects. 
	\end{tablenotes}
	\end{threeparttable}
	\label{tab5.1}
\end{table}

We report the baseline estimates of import exchange rate pass-through in panel A. Column (1) shows the import exchange rate pass-through using equation (1) for the long sample from 2000 to 2011, including all companies that appear at least once in customs records, whether or not they are registered in the CIE database. Column (2) shows the results for import exchange rate pass-through for the matched sample (importers registered in the CIE database from 2000 to 2007). The pass-through coefficient in column (2) is larger than the one in column (1), yet both are incomplete. The average import exchange rate pass-through for China in the long sample is around 18\%, while in the matched sample about 39\%. The latter ERPT means that the import prices denominated in RMB will increase by about 3.9\% during a 10\% real depreciation and decrease by the same amount during a 10\% appreciation. Column (3) and column (4) report the import pass-through among a subset of the matched sample including only the top 50 and top 20 trading partners by import value. The results in those subsamples of top partners are slightly less complete than those in column (2), yet the levels are of similar magnitude. Import price fluctuations will reflect nearly one-third of exchange rate fluctuations, with the remainder absorbed by foreign currency price fluctuations. These results imply that Chinese importers have to bear less than half but still considerable cost fluctuations due to exchange rate shocks. 

Accordingly, estimates of export exchange rate pass-through are recorded in panel B. Similar to which in panel A for imports, column (1) and column (2) in panel B show the export exchange pass-through using the long sample and the matched sample. Column (3) and column (4), in turn, show the export price pass-through to major export destinations. The estimated export pass-through in each column equals one minus the coefficient at the row of $\Delta lnRER_{ct}$. The export pass-through ranges from 93.3\% for the top 20 countries to 96.6\% for the matched sample, which is near complete. Specifically, 95\% export ERPT means that a 10\% real appreciation of RMB concerning a destination market is associated with a 0.5\% decrease in the RMB price while a 9.5\% increase in foreign currency price. That is to say, export price in RMB adjusts very little with exchange rate fluctuations.

The strikingly almost complete ERPT into RMB export price echoes the finding of LMX\cite{lmx2015}. One explanation could be that most Chinese exporters are at the lower end of the value chain, where high ERPT exposes their low profit margins, leaving no room for pricing-to-market strategy. Most Chinese exporters have no choice but to pass all exchange rate swings to their destination prices, regardless of potential better monopolistic competition strategies. In addition, since RMB is pegged to the dollar until 2005, the complete pass-through to price in destination currency is consistent with a rigid price model with the US dollar as the invoice currency, at least in those countries with freely floating exchange rates relative to the U.S. dollar. We will later provide results using samples excluding the US and other countries whose currency is pegged to the US dollar.

From the comparison, it can be seen that the exchange rate-price pass-through on China's import side is much less complete than that on the export side. That is to say, for Chinese firms, when the RMB depreciates against the currencies of major trading partners, export prices denominated in RMB will not rise significantly, but their import costs will rise sharply; on the contrary; when the real exchange rate of RMB appreciates, export prices in RMB will decrease only to a limited extent, and their import costs will increase dramatically drop. If we consider a typical two-way trader in China who simultaneously imports and exports from two groups of countries with strong correlations in exchange rate fluctuations, a devaluation of the local currency will reduce his unit profits, while an appreciation of the local currency will widen his profit margins. We call this phenomenon the asymmetry of the import and export exchange rate pass-through, which exposes Chinese trading companies to two-way exchange rate risks.

\section{Effects of Credit Constraints}\label{sec-5.2}

Another goal of our paper is to assess how importers with varying degrees of financial vulnerability absorb exchange rate fluctuations when the home currency depreciates or appreciates. We evaluate the consequences of credit constraints on the firms' price responses to exchange rate shocks using equation \ref{eq4.2} in section \ref{seq-4.1.2}. Table \ref{tab5.2} presents differences in exchange rate pass-through into import prices and export resulting from the industry-level credit demand heterogeneity. Panel A reports the results for credit constraints and import pass-through and panel B reports the comparing results for the export side. 

\begin{table}[htbp]
	\centering
	\caption{Effects of Credit Constraints on Exchange Rate Pass-Through}
	\begin{threeparttable}	
	\begin{tabular}{lcccc}
		\toprule
		& (1)   & (2)   & (3)   & (4) \\
		\midrule
		Panel A & \multicolumn{4}{c}{Import} \\
		& FPC   & External Finance & Tangibility & Inventory \\
		\midrule
		$\Delta \ln RER_{ct}$ & 0.123*** & 0.218*** & 1.175*** & -0.739*** \\
		& (0.016) & (0.016) & (0.033) & (0.069) \\
		$\Delta \ln RGDR_{ct}$ & 0.314*** & 0.323*** & 0.283*** & 0.273*** \\
		& (0.090) & (0.090) & (0.090) & (0.090) \\
		$\Delta \ln RER_{ct}*FPC_{j}$ & 0.379*** &       &       &  \\
		& (0.010) &       &       &  \\
		$\Delta \ln RER_{ct}*ExtFin_{j}$ &       & 1.159*** &       &  \\
		&       & (0.029) &       &  \\
		$\Delta \ln RER_{ct}*Tang_{j}$ &       &       & -3.305*** &  \\
		&       &       & (0.117) &  \\
		$\Delta \ln RER_{ct}*Invent_{j}$ &       &       &       & 6.305*** \\
		&       &       &       & (0.389) \\
		Year FE  & Yes   & Yes   & Yes   & Yes \\
		Firm-product-country FE & Yes   & Yes   & Yes   & Yes \\
		Observations & 1792020 & 1792020 & 1792020 & 1792020 \\
		\midrule
		Panel B & \multicolumn{4}{c}{Export} \\
		& FPC   & External Finance & Tangibility & Inventory \\
		\midrule
		$\Delta \ln RER_{ct}$ & 0.039*** & 0.034*** & -0.030** & 0.102*** \\
		& (0.006) & (0.005) & (0.015) & (0.030) \\
		$\Delta \ln RGDR_{ct}$ & -0.084** & -0.083** & -0.084** & -0.083** \\
		& (0.037) & (0.037) & (0.037) & (0.037) \\
		$\Delta \ln RER_{ct}*FPC_{j}$ & -0.019*** &       &       &  \\
		& (0.004) &       &       &  \\
		$\Delta \ln RER_{ct}*ExtFin_{j}$ &       & -0.045*** &       &  \\
		&       & (0.013) &       &  \\
		$\Delta \ln RER_{ct}*Tang_{j}$ &       &       & 0.230*** &  \\
		&       &       & (0.053) &  \\
		$\Delta \ln RER_{ct}*Invent_{j}$ &       &       &       & -0.412** \\
		&       &       &       & (0.171) \\
		Year FE  & Yes   & Yes   & Yes   & Yes \\
		Firm-product-country FE & Yes   & Yes   & Yes   & Yes \\
		Observations & 1793974 & 1793974 & 1793974 & 1793974 \\
		\bottomrule
	\end{tabular}
	\label{tab5.2}
	\begin{tablenotes}
		\footnotesize
		\item[*] Standard errors in parentheses; *, **, and *** indicate significance at 10\%, 5\% and 1\% levels. The dependent variable is the price change $\Delta \ln P_{ijct}$. Columns (1)-(4) use different measures of credit constraints calculated using U.S. data. Panel A shows the results of import ERPT while panel B shows the results of export ERPT. All regressions include firm-product-country fixed effects and year fixed effects.
	\end{tablenotes}
	\end{threeparttable}
\end{table}

We are particularly interested in the coefficients of the interaction terms. Note that a larger elasticity coefficient on the import side implies a more complete exchange rate-price pass-through, and similarly a positive cross-term coefficient implies that the magnitude of this variable is positively related to exchange rate pass-through. Using the first principal component of external finance dependence and asset tangibility $FPC_j$ to measure financial vulnerability $FV_j$, we see that import exchange rate pass-through is more complete in financially more vulnerable sectors, relative to financially less vulnerable sectors (column 1, row 3). Columns (2) and (3) separately show the effects of external finance dependence and asset tangibility on importers' exchange rate pass-through. Consistent with the definition that higher external finance dependence implies tighter credit constraints faced by firms while higher asset tangibility can alleviate them, we observe a positive coefficient for the former (row 5) and a negative coefficient for the latter (row 6). When we use the auxiliary measure $Invent_j$, we further observe that the effect on exchange rate pass-through is positive (column 4). Overall, the coefficient $\beta^{Import}_2$ of the interaction term $\Delta \ln RER_{ct} \cdot FV_{j}$ is positive and significant at the 1\% level. Our evidence supports the intuition that exchange rate fluctuations are more likely to be reflected in unstable import costs for importers in more financially vulnerable industries because they have weak bargaining power in the international market. External financing dependence, internal collateral capacity, and inventory turnover all act jointly and separately on the exchange rate pass-through of importers.

It is worth noting that the exchange rate pass-through here is estimated by the at-the-dock import price in our specification. This excludes any impact of credit constraints on post-landing costs, such as local distribution and logistics costs. In other words, credit constraints imposed on Chinese importers can affect home import costs by directly affecting the supplier's pricing behavior. Intuitively, firms in more credit-constrained industries may have weaker import bargaining power, and products are more likely to be pegged to the dollar or the currency of the exporting country (less pricing-to-market), so at-the-dock prices are more affected by exchange rate fluctuations, regardless of any domestic market factor.

By substituting the superscript D in equation \ref{eq4.2} to export, we obtain a comparative result of the effect of credit constraints on export price pass-through. Estimates in columns (1), (2), and (4) all show significantly negative coefficients on interaction terms while column (3) shows a negative significant coefficient. The estimates suggest that financial constraints lead export exchange rate pass-through to a more complete degree, although the original result is already close to complete. These results verified the conclusion of \cite{strasser2013} who argues that financially constrained firms have higher export price pass-through compared to unconstrained firms. That is to say, credit constraints restrict exporters from absorbing exchange rate shocks, potentially because firms need external finance to apply pricing-to-market strategies in foreign markets.

Comparing panel A and panel B, although the import ERPT is still less complete than the export ERPT, we can reach a consistent conclusion that credit constraints steer both of them toward a more complete direction. Following the analysis in section \ref{sec-5.1}, credit constraints expose Chinese manufacturing firms to greater exchange rate risk in international trade. Exporters with more vulnerable credit are forced to lower destination prices when RMB depreciates compared to those with unrestricted credit, while RMB income remained relatively unchanged, and importers’ costs rose more significantly; in contrast, when RMB appreciates, restricted exporters will increase destination prices more, even if it means losing their competitive advantages, and importers' costs will be reduced at this time. For credit-constrained two-way traders, given import sources and export markets cannot be adjusted quickly, the unit profit margin is more sensitive to exchange rate fluctuations.

Nonetheless, the direction in which credit constraints affect the exchange rate pass-through on the export side and the import side is the same, the underlying channels may work differently. Following \cite{strasser2013}, a higher external finance premium causes higher marginal costs. Thus, firms with binding financial constraints have no choice but set higher prices and face a higher price elasticity of demand. When there is an exchange rate shock, the optimal choice is to adjust their markups but credit-constrained firms can do so only to a limited extent because they have narrower profit margins. However, for import ERPT, credit constraints affect how buyers pay in the transactions. Adequate credit or cash reserves give importers a better bargaining chip, for example, by allowing them to negotiate longer-term purchase agreements, where exchange rate fluctuations will be more borne by international sellers. In contrast, a credit-strapped importer may not have the buffers to transfer risk, so it must accept current exchange rate settlements, although that means taking on more volatile prices.

\section{Sourcing Diversity and Credit Constraints}

In this section, we further study the factors that directly affect the purchasing side. As emerged in an intuitive guess, a potential mechanism through which financial constraints affect an importer's bargaining power with foreign suppliers is its outside sourcing options. Companies with more trading partners can flexibly adjust the weight of imports from different countries. Firms with heterogeneous sourcing capacity may thus be affected by credit constraints to a different extent. 

Therefore, we employ equation \ref{eq4.4} to include the number of import sources described in section \ref{sec-4.2.3}. The estimation results are reported in the below Table \ref{tab5.3} and confirm the empirical relevance of differences in sourcing diversity across firms. Similarly, results for sales destination diversity on the export side are provided in panel B \ref{tabA.1} as a comparison.

\begin{table}[htbp]
	\centering
	\caption{Import Sources and Effects of Credit Constraints on Import Exchange Rate Pass-Through}
	\begin{threeparttable}
	\begin{tabular}{lcccc}
		\toprule
		& (1)   & (2)   & (3)   & (4)    \\
		\midrule
		Panel A & \multicolumn{4}{c}{Import} \\
		 & \#Sources & \#Sources+ & \#Sources+ & \#Sources+	 \\
		&       & FPC &External & Tangibility \\
		&       & &Finance &			\\
		\midrule
		$\Delta \ln RER_{ct}$ & 0.433*** & 0.177*** & 0.274*** & 1.386*** \\
		& (0.017) & (0.019) & (0.018) & (0.040) \\
		$\Delta \ln RGDR_{ct}$ & 0.250*** & 0.292*** & 0.297*** & 0.267*** \\
		& (0.090) & (0.090) & (0.090) & (0.090) \\
		$\#Source_{ijt}$ & -0.021*** & -0.016*** & -0.016*** & -0.050*** \\
		& (0.002) & (0.003) & (0.003) & (0.006) \\
		$\Delta \ln RER_{ct}*FPC_{j}*\#Source_{ijt}$ &       & -0.014*** &       &  \\
		&       & (0.002) &       &  \\
		$\Delta \ln RER_{ct}*FPC_{j}$ &       & 0.443*** &       &  \\
		&       & (0.012) &       &  \\
		$\Delta \ln RER_{ct}*ExtFin_{j}*\#Source_{ijt}$ &       &       & -0.054*** &  \\
		&       &       & (0.006) &  \\
		$\Delta \ln RER_{ct}*ExtFin_{j}$ &       &       & 1.410*** &  \\
		&       &       & (0.037) &  \\
		$\Delta \ln RER_{ct}*Tang_{j}*\#Source_{ijt}$ &       &       &       & 0.104*** \\
		&       &       &       & (0.026) \\
		$\Delta \ln RER_{ct}*Tang_{j}$ &       &       &       & -3.790*** \\
		&       &       &       & (0.148) \\
		Year FE  & Yes   & Yes   & Yes   & Yes \\
		Firm-product-country FE & Yes   & Yes   & Yes   & Yes \\
		Observations & 1792020 & 1792020 & 1792020 & 1792020 \\
		\bottomrule
	\end{tabular}
	\begin{tablenotes}
	\footnotesize
	\item[*] Standard errors in parentheses; *, **, and *** indicate significance at 10\%, 5\% and 1\% levels. The dependent variable is the price change $\Delta \ln P_{ijct}$. Columns (2)-(4) use different measures of credit constraints calculated using U.S. data. Panel A shows the results of import ERPT while panel B shows the results of export ERPT. Panel B is shown in Appendix \ref{tabA.1}. All regressions include firm-product-country fixed effects and year fixed effects.
	\end{tablenotes}
	\end{threeparttable}
	\label{tab5.3}
\end{table}

The estimates for intersection terms between import sources and real exchange rate changes are displayed in column (1). We find that importers who import a certain product from more sources will have a less complete pass-through. This is consistent with our hypothesis that importers with more alternative sourcing options will have less complete pass-through. Interestingly, exporters who export to more destinations (both for a certain product or in total) will have a slightly more complete pass-through. In other words, the diversity of import sources for the same product can significantly enhance the stability of import prices, but the diversity of export markets does not.

In columns (2)-(4), after adding interactions, we find the effects of credit constraints still exist while the triple interaction terms with the number of sources have the opposite and significant coefficients. That means a wider sourcing base will mitigate the effects of credit constraints, in addition to its effect on pass-through. We continue to observe that this triple interaction effect only works for the import side in panel A but not for the export side in panel B (attached in appendix \ref{tabA.1}). The opposing effects of credit constraints and purchasing diversity on exchange rate pass-through confirm our conjecture about the bargaining power of importers. If a firm can import the same product from more sources, it has more flexibility in the face of bilateral exchange rate shocks in individual markets. In other words, constrained firms with more import sources have more ways to escape the unfavorable exchange rate risk. A more diverse importer can either switch from one source to another to reduce costs (trade diversion effect) or make a more credible threat to negotiate a more stable price. 