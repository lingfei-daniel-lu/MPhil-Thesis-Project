\chapter{Robustness}\label{sec-6.robustness}

In this chapter, we further check the robustness of our main results with alternative measures and sub-samples.

\section{Alternative Measures of Credit Constraints}

As the first robustness test to verify our baseline results, we use alternative credit constraint measures from CIE database. The purpose is to avoid potential bias from differences in the attributes of industry credit demand in different countries. The details of constructing these Chinese variables are discussed in in section \ref{sec-4.2.2}. Although Chinese financial market is less mature than that of the US, the rankings of industries in credit constraints are comparable. Thus, the results based on the credit constraints measures from Chinese data are expected be consistent with our main findings. Our results are reported in Table \ref{tab7.1} which can be easily compared with the results using US measures.

\begin{table}[htbp]
	\centering
	\caption{Alternative Estimations with Chinese Measures of Credit Constraints}
	\begin{threeparttable}
	\begin{tabular}{lcccc}
		\toprule
		& (1)   & (2)   & (3)   & (4) \\
		\midrule
		Panel A & \multicolumn{4}{c}{Import} \\
		& External Finance & Tangibility & Inventory & R\&D Intensity \\
		\midrule
		$\Delta \ln RER_{ct}$ & 0.502*** & 2.032*** & -0.752*** & 0.043** \\
		& (0.018) & (0.053) & (0.043) & (0.019) \\
		$\Delta \ln RGDP_{ct}$ & 0.249*** & 0.240*** & 0.333*** & 0.304*** \\
		& (0.090) & (0.090) & (0.090) & (0.090) \\
		$\Delta \ln RER_{ct}$*$ExtFin_{j}$ & 0.223*** &       &       &  \\
		& (0.014) &       &       &  \\
		$\Delta \ln RER_{ct}$*$Tang_{j}$ &       & -5.776*** &       &  \\
		&       & (0.176) &       &  \\
		$\Delta \ln RER_{ct}*Invent_{j}$ &       &       & 9.797*** &  \\
		&       &       & (0.352) &  \\
		$\Delta \ln RER_{ct}*R\&D_{j}$ &       &       &       & 16.398*** \\
		&       &       &       & (0.573) \\
		Year FE  & Yes   & Yes   & Yes   & Yes \\
		Firm-product-country FE & Yes   & Yes   & Yes   & Yes \\
		Observations & 1792020 & 1792020 & 1792020 & 1792020 \\
		\midrule
		Panel B & \multicolumn{4}{c}{Export} \\
		& External Finance & Tangibility & Inventory & R\&D Intensity \\
		\midrule
		$\Delta \ln RER_{ct}$ & 0.016** & -0.135*** & 0.086*** & 0.043*** \\
		& (0.007) & (0.023) & (0.019) & (0.007) \\
		$\Delta \ln RGDP_{ct}$ & -0.081** & -0.082** & -0.081** & -0.082** \\
		& (0.037) & (0.037) & (0.037) & (0.037) \\
		$\Delta \ln RER_{ct}$*$ExtFin_{j}$ & -0.021*** &       &       &  \\
		& (0.007) &       &       &  \\
		$\Delta \ln RER_{ct}$*$Tang_{j}$ &       & 0.557*** &       &  \\
		&       & (0.074) &       &  \\
		$\Delta \ln RER_{ct}*Invent_{j}$ &       &       & -0.504*** &  \\
		&       &       & (0.165) &  \\
		$\Delta \ln RER_{ct}*R\&D_{j}$ &       &       &       & -0.627*** \\
		&       &       &       & (0.243) \\
		Year FE  & Yes   & Yes   & Yes   & Yes \\
		Firm-product-country FE & Yes   & Yes   & Yes   & Yes \\
		Observations & 1793974 & 1793974 & 1793974 & 1793974 \\
		\bottomrule
	\end{tabular}
	\label{tab6.1}
	\begin{tablenotes}
		\footnotesize
		\item[*]  Standard errors in parentheses; *, **, and *** indicate significance at 10\%, 5\% and 1\% levels. The dependent variable is the price change $\Delta \ln P_{ijct}$. Columns (1)-(4) use different measures of credit constraints calculated using Chinese data. Panel A shows the results of import ERPT while panel B shows the results of expor,t ERPT. All regressions include firm-product-country fixed effects and year fixed effects.
	\end{tablenotes}
	\end{threeparttable}
\end{table}

Columns 1-4 of panel A of table \ref{tab6.1} present the import-side results for external finance dependence, tangibility, inventory ratio, and R\&D intensity, respectively. Panel B reports the export-side results with the same variables. All regressions include firm-product-country fixed effects and year fixed effects as before. Nevertheless, most of the interaction term coefficients exhibit the same signs as above, confirming the validity of our baseline findings of the effects of credit constraints on exchange rate pass-through. We can still conclude that financially more constrained firms (both importers and exporters) have more complete exchange rate pass-through than those less constrained, even with Chinese measures.

\section{Sub-sample Test: Two-way traders}

Since most of China's imports go through two-way traders, who conduct both export and import simultaneously, import exchange rate pass-through is likely to be related to export behaviors. Specifically, importers who also export may pass part of the price fluctuations of imported intermediate goods caused by exchange rate shocks to their export destination to buffer the impact of exchange rate risks. To test whether the pattern of China's import pass-through still holds for two-way traders, we use the sub-sample consisting only of two-way traders for a robustness check. Table \ref{tab6.2} presents the results for this restricted sample.

\begin{table}[htbp]
	\centering
	\caption{Alternative Estimations with Two-way Traders}
	\begin{threeparttable}
	\begin{tabular}{lcccc}
		\toprule
		& (1)   & (2)   & (3)   & (4) \\
		\midrule
		Panel A & \multicolumn{4}{c}{Import (Two-way traders)} \\
		& Baseline & FPC   & External Finance & Tangibility \\
		\midrule
		$\Delta \ln RER_{ct}$ & 0.394*** & 0.136*** & 0.231*** & 1.158*** \\
		& (0.015) & (0.016) & (0.015) & (0.031) \\
		$\Delta \ln RGDP_{ct}$ & 0.406*** & 0.459*** & 0.469*** & 0.427*** \\
		& (0.086) & (0.086) & (0.086) & (0.086) \\
		$\Delta \ln RER_{ct}$*$FPC_{j}$ &       & 0.388*** &       &  \\
		&       & (0.009) &       &  \\
		$\Delta \ln RER_{ct}$*$ExtFin_{j}$ &       &       & 1.246*** &  \\
		&       &       & (0.028) &  \\
		$\Delta \ln RER_{ct}$*$Tang_{j}$ &       &       &       & -3.138*** \\
		&       &       &       & (0.112) \\
		Year FE  &       & Yes   & Yes   & Yes \\
		Firm-product-country FE &       & Yes   & Yes   & Yes \\
		Observations & 1712289 & 1712289 & 1712289 & 1712289 \\
		\midrule
		Panel B & \multicolumn{4}{c}{Export (Two-way traders)} \\
		& Baseline & FPC   & External Finance & Tangibility \\
		\midrule
		$\Delta \ln RER_{ct}$ & 0.040*** & 0.051*** & 0.044*** & -0.034** \\
		& (0.006) & (0.006) & (0.006) & (0.016) \\
		$\Delta \ln RGDP_{ct}$ & -0.144*** & -0.145*** & -0.144*** & -0.147*** \\
		& (0.041) & (0.041) & (0.041) & (0.041) \\
		$\Delta \ln RER_{ct}$*$FPC_{j}$ &       & -0.022*** &       &  \\
		&       & (0.005) &       &  \\
		$\Delta \ln RER_{ct}$*$ExtFin_{j}$ &       &       & -0.048*** &  \\
		&       &       & (0.015) &  \\
		$\Delta \ln RER_{ct}$*$Tang_{j}$ &       &       &       & 0.284*** \\
		&       &       &       & (0.059) \\
		Year FE  &       & Yes   & Yes   & Yes \\
		Firm-product-country FE &       & Yes   & Yes   & Yes \\
		Observations & 1415415 & 1415415 & 1415415 & 1415415 \\
		\bottomrule
	\end{tabular}
	\label{tab6.2}
	\begin{tablenotes}
		\footnotesize
		\item[*] Standard errors in parentheses; *, **, and *** indicate significance at 10\%, 5\% and 1\% levels. The dependent variable is the price change $\Delta \ln P_{ijct}$. We only include those firms that both import and export during the same year. Columns (2)-(4) use different measures of credit constraints calculated using U.S. data. Panel A shows the results of import ERPT while panel B shows the results of export ERPT. All regressions include firm-product-country fixed effects and year fixed effects.
	\end{tablenotes}
	\end{threeparttable}
\end{table}

Table \ref{tab6.2} shows that the results for the subset of two-way traders are highly similar to the results for the entire matched sample, indicating that the exchange rate pass-through pattern of two-way traders dominates China’s imports and exports. The results from the sample of two-way traders are still typical, whether estimating the exchange rate pass-through of imports and exports or examining the role of credit constraints. Combined with our discussion later, the nature of two-way traders should be the focus of future research on import-export connections.

\section{Sub-sample Test: Excluding US Dollar Peg}

Due to China's exchange rate peg to the US dollar until July 2005, we observed very weak movements in the exchange rate of RMB against the US dollar and other currencies pegged to the US dollar as shown in \ref{fig3.1} and \ref{fig3.2}. Therefore, in this test, we exclude the US and other countries that use the US dollar as their official currency or whose currency is pegged to the US dollar from our baseline matched sample. \footnote{The excluded countries and regions are Belize, Liberia, Qatar, Ecuador, Djibouti, Saint Lucia, Saint Kitts and Nevis, St. Vincent and the Grenadines, Dominica, Antigua and Barbuda, Palestine, Bahamas, Barbados, Panama, Bahrain, Curaçao, Grenada, Saudi Arabia, Zimbabwe, Macao SAR of China, Turks and Caicos Islands, Bermuda, Jordan, United States, British Virgin Islands, Dutch Sint Maarten, El Salvador, Montserrat, United Arab Emirates, Oman, Aruba, Hong Kong SAR of China, Maldives, Lebanon.}. We examine whether the previous main results are robust after excluding those countries whose currencies are denominated in or pegged to the US dollar.

\begin{table}[htbp]
	\centering
	\caption{Alternative Estimations Excluding US Dollar-pegged countries}
	\begin{threeparttable}
	\begin{tabular}{lllll}
		\toprule
		& (1)   & (2)   & (3)   & (4) \\
		\midrule
		Panel A & \multicolumn{4}{c}{Import (Excluding US Dollar Peg)} \\
		& Baseline & FPC   & External Finance & Tangibility \\
		\midrule
		$\Delta \ln RER_{ct}$ & 0.382*** & 0.164*** & 0.256*** & 1.173*** \\
		& (0.017) & (0.018) & (0.018) & (0.035) \\
		$\Delta \ln RGDP_{ct}$ & 0.709*** & 0.778*** & 0.797*** & 0.730*** \\
		& (0.106) & (0.106) & (0.106) & (0.106) \\
		$\Delta \ln RER_{ct}$*$FPC_{j}$ &       & 0.362*** &       &  \\
		&       & (0.010) &       &  \\
		$\Delta \ln RER_{ct}$*$ExtFin_{j}$ &       &       & 1.103*** &  \\
		&       &       & (0.030) &  \\
		$\Delta \ln RER_{ct}$*$Tang_{j}$ &       &       &       & -3.186*** \\
		&       &       &       & (0.121) \\
		Year FE  &       & Yes   & Yes   & Yes \\
		Firm-product-country FE &       & Yes   & Yes   & Yes \\
		Observations & 1489984 & 1489984 & 1489984 & 1489984 \\
		\midrule
		Panel B & \multicolumn{4}{c}{Export (Excluding US Dollar Peg)} \\
		& Baseline & FPC   & External Finance & Tangibility \\
		\midrule
		$\Delta \ln RER_{ct}$ & 0.031*** & 0.038*** & 0.034*** & -0.021 \\
		& (0.006) & (0.006) & (0.006) & (0.016) \\
		$\Delta \ln RGDP_{ct}$ & -0.023 & -0.024 & -0.023 & -0.025 \\
		& (0.049) & (0.049) & (0.049) & (0.049) \\
		$\Delta \ln RER_{ct}$*$FPC_{j}$ &       & -0.016*** &       &  \\
		&       & (0.005) &       &  \\
		$\Delta \ln RER_{ct}$*$ExtFin_{j}$ &       &       & -0.039*** &  \\
		&       &       & (0.014) &  \\
		$\Delta \ln RER_{ct}$*$Tang_{j}$ &       &       &       & 0.196*** \\
		&       &       &       & (0.055) \\
		Year FE  &       & Yes   & Yes   & Yes \\
		Firm-product-country FE &       & Yes   & Yes   & Yes \\
		Observations & 1371760 & 1371760 & 1371760 & 1371760 \\
		\bottomrule
	\end{tabular}
		\label{tab6.3}
	\begin{tablenotes}
		\footnotesize
		\item[*] Standard errors in parentheses; *, **, and *** indicate significance at 10\%, 5\% and 1\% levels. The dependent variable is the price change $\Delta \ln P_{ijct}$. We do not include countries that use the U.S. dollar or currencies pegged to the U.S. dollar. Columns (2)-(4) use different measures of credit constraints calculated using U.S. data. Panel A shows the results of import ERPT while panel B shows the results of export ERPT. All regressions include firm-product-country fixed effects and year fixed effects.
	\end{tablenotes}
\end{threeparttable}
\end{table}

Table \ref{tab6.3} demonstrates that the results without the U.S. and those pegged countries are not overturned. The estimated import ERPT for this subsample is 38.2\%, which is slightly more complete than 35.7\% in the matched sample shown in Table \ref{tab5.1}. In turn, the estimated export ERPT for this subsample is 96.0\%, which is less complete than 96.9\% in the matched sample. All coefficients on interaction terms remain robust and significant, suggesting that higher external finance dependence makes exchange rate pass-through more complete while higher asset tangibility plays an opposite role. Therefore, the peg of RMB to the dollar before 2005 did not interfere with the main conclusions of this paper.