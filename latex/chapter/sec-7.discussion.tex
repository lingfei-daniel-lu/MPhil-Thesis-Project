\chapter{More Discussions}\label{sec-7.discussion}

\section{Firm Heterogeneity in Markup}

Given that credit constraints play an important role in China's import exchange rate pass-through, we proceed to analyze how different factors participate in determining import exchange rate pass-through, and whether they can replace credit constraints or not. One major argument in the literature is that firms with different sales markup may have heterogeneous responses to exchange rate shocks as suggested by \cite{bmm2012} and \cite{lmx2015}. The same logic could also apply to import exchange rate pass-through. Besides, \cite{llz2018} provide microeconomic evidence that both internal finance and external credit supply significantly promote firms' sales growth rates. Following the empirical framework in section \ref{sec-4.1.3}, we add estimated firm-level markup into interactions, and the results are shown in Table \ref{tab7.1}.

\begin{table}[htbp]
	\centering
	\caption{Heterogeneous Markup and Effects of Credit Constraints on Import Exchange Rate Pass-through}
	\begin{threeparttable}
		\begin{tabular}{lcccc}
			\midrule          & (1)   & (2)   & (3)   & (4)     \\
			\midrule
			Panel A & \multicolumn{4}{c}{Import} \\
			& Markup & Markup+& Markup+ & Markup+      \\
			&       &FPC& External & Tangibility        \\
			&       & & Finance &  	          \\
			\midrule
			$\Delta \ln RER_{ct}$ & 0.459*** & 0.310*** & 0.431*** & 1.419***   \\
			& (0.045) & (0.045) & (0.045) & (0.058) \\
			$\Delta \ln RGDP_{ct}$ & 0.329*** & 0.376*** & 0.389*** & 0.343*** \\
			& (0.106) & (0.106) & (0.106) & (0.106)  \\
			$\Delta \ln RER_{ct}$*$Markup_{jt-1}$ & -0.073** & -0.157*** & -0.163*** & -0.112*** \\
			& (0.030) & (0.030) & (0.031) & (0.030) \\
			$\Delta \ln RER_{ct}$*$FPC_{j}$ &       & 0.414*** &       &  \\
			&       & (0.011) &       &   \\
			$\Delta \ln RER_{ct}$*$ExtFin_{j}$ &       &       & 1.232*** &  \\
			&       &       & (0.034) &   \\
			$\Delta \ln RER_{ct}$*$Tang_{j}$  &       &       &       & -3.692*** \\
			&       &       &       & (0.139) \\
			$Markup_{jt-1}$ & 0.013** & 0.009 & 0.007 & 0.012**  \\
			& (0.006) & (0.006) & (0.006) & (0.006)  \\
			Year FE  & Yes   & Yes   & Yes   & Yes       \\
			Firm-product-country FE & Yes   & Yes   & Yes   & Yes       \\
			Observations & 1411106 & 1411106 & 1411106 & 1411106  \\
			\bottomrule
		\end{tabular}
		\begin{tablenotes}
			\footnotesize
			\item[*] Standard errors in parentheses; *, **, and *** indicate significance at 10\%, 5\% and 1\%. The dependent variable is the price change $\Delta \ln P_{ijct}$. Columns (2)-(4) use different measures of credit constraints calculated using U.S. data. Panel A shows the effects of markup and credit constraints on import ERPT while panel B shows their effects on export ERPT. Panel B is shown in Appendix \ref{tabA.2}. All regressions include firm-product-country fixed effects and year fixed effects.
		\end{tablenotes}
	\end{threeparttable}
	\label{tab7.1}
\end{table}

Panel A shows how markup affects import ERPT in addition to credit constraints and panel B (attached in \ref{tabA.2}) shows the comparison results for export ERPT. Column (1) shows the direct effects of firm-level markup on exchange rate pass-through. Columns (2) to (4) add the first principal component, external finance dependence, and asset tangibility to the regression respectively. All columns control for the lag term of markup.

From the coefficients of interaction terms in panel A, firms with higher markup have less complete import pass-through while firms. The effects of financial constraints are still significant and robust as in section \ref{sec-5.2} after controlling for markup. The coefficients in panel B \ref{tabA.2} imply a similar conclusion. Exporters with higher markup have less complete export pass-through. In other words, higher markup on the exchange rate pass-through work in the opposite direction with tighter credit constraints. However, the coefficients in rows (4)-(6) are significant respectively, indicating that markup cannot fully explain the effect of credit constraints, which is consistent with \cite{xu-guo2021}.

The explanation for import price absorptive capacity seems more complicated than which for export ERPT. BMM\cite{bmm2012} documents that more productive firms react to depreciation (or appreciation) by adjusting more markup and less export volume, keeping local market prices relatively stable, which means a less complete pass-through. This explanation hinges on endogenous markup over marginal costs where less elastic demand allows them to adjust markups more extensively during currency fluctuations. However, on the import side, other factors influence the sourcing capacity upon exchange rate movements. In any case, the effects of credit constraints are not offset or replaced by the sales factors, so the conclusions in section \ref{sec-5.2} about credit constraints remain valid.

\section{Firm Heterogeneity in Market Share}

In this section, we first provide the regression results of equation \ref{eq4.5} with the market share and its square term constructed in section \ref{sec-4.2.5}. The results are presented in Table \ref{tab7.2}. The coefficient estimates for $\beta_3$ and $\beta_4$ can be used to map out the ERPT–import market share relationship.

\begin{table}[htbp]
	\centering
	\caption{Market Share and Effects of Credit Constraints on Import Exchange Rate Pass-through}
	\begin{threeparttable}
		\begin{tabular}{lccccc}
			\toprule
			& (1)   & (2)   & (3)   & (4) & (5)\\
			\midrule
			Panel A & \multicolumn{5}{c}{Import} \\
			& MS    & $MS^2$ & FPC   & External & Tangibility \\
			&       &       &       & Finance &  \\
			\midrule
			$\Delta \ln RER_{ct}$ & 0.392*** & 0.381*** & 0.119*** & 0.220*** & 1.175*** \\
			& (0.016) & (0.016) & (0.018) & (0.017) & (0.033) \\
			$\Delta \ln RGDP_{ct}$ & 0.247*** & 0.251*** & 0.314*** & 0.321*** & 0.280*** \\
			& (0.090) & (0.090) & (0.090) & (0.090) & (0.090) \\
			$\Delta \ln RGDP_{ct}*MS_{ijct}$ & -0.305*** & 0.155 & 0.677*** & 0.556*** & 0.546*** \\
			& (0.042) & (0.156) & (0.156) & (0.156) & (0.156) \\
			$\Delta \ln RGDP_{ct}*MS^2_{ijct}$ &       & -0.531*** & -0.941*** & -0.837*** & -0.846*** \\
			&       & (0.173) & (0.173) & (0.173) & (0.173) \\
			$\Delta \ln RER_{ct}$*$FPC_{j}$ &       &       & 0.379*** &       &  \\
			&       &       & (0.010) &       &  \\
			$\Delta \ln RER_{ct}$*$ExtFin_{j}$ &       &       &       & 1.156*** &  \\
			&       &       &       & (0.029) &  \\
			$\Delta \ln RER_{ct}$*$Tang_{j}$ &       &       &       &       & -3.290*** \\
			&       &       &       &       & (0.118) \\
			$MS_{ijpt}$    & -0.012 & -0.009 & -0.002 & -0.004 & -0.003 \\
			& (0.012) & (0.012) & (0.012) & (0.012) & (0.012) \\
			Year FE  & Yes   & Yes   & Yes   & Yes   & Yes \\
			Firm-product-country FE & Yes   & Yes   & Yes   & Yes   & Yes \\
			Observations & 1792020 & 1792020 & 1792020 & 1792020 & 1792020 \\
			\bottomrule
		\end{tabular}
		\begin{tablenotes}
			\footnotesize
			\item[*] Standard errors in parentheses; *, **, and *** indicate significance at 10\%, 5\%, and 1\%. The dependent variable is the price change $\Delta \ln P_{ijct}$. Columns (3)-(5) use different measures of credit constraints calculated using U.S. data. Panel A shows the effects of market share and credit constraints on import ERPT while panel B shows their effects on export ERPT. Panel B is shown in Appendix \ref{tabA.3}. All regressions include firm-product-country fixed effects and year fixed effects.
		\end{tablenotes}
	\end{threeparttable}
	\label{tab7.2}
\end{table}

In columns (1) and (2), we add the primary and quadratic terms of market share to the baseline estimation of ERPT in turn. In columns (3)-(4), we further include the effects of external finance dependence and asset tangibility on top of column (2). Panel B shows the results for the export side. We find that there is evidence of a negative relationship between import pass-through and market share; however, the coefficient for the linear interaction term is positive, while the strong negative effect lies on the squared interaction term, suggesting some curvature in this relationship. The effect of market share on export ERPT is not significant. 

In addition, we also perform group regressions by market share quartile and report the results in Table \ref{tab7.2}. Columns (1)-(4) show the import exchange rate pass-through for importers within each quartile of the market share distribution (0-25\%, 25\%-50\%, 50\%-75\%, 75\%-100\%), respectively. Panel B in turn shows the results for exporters within different quartiles in the market share distribution. Results for each quartile group with interaction terms of credit constraints are not provided here but are available upon request. Roughly speaking, the import exchange rate pass-through and market share show a hump-shaped (inverted U-shaped) relationship, and the export exchange rate pass-through coefficient (weakly) decreases with the market share (towards complete price pass-through in upper quartiles).

\begin{table}[htbp]
	\centering
	\caption{Estimations of Exchange Rate Pass-Through by Market Share Quartile}
	\begin{threeparttable}
		\begin{tabular}{lcccc}
			\toprule
			& (1)   & (2)   & (3)   & (4) \\
			\midrule
			Panel A & \multicolumn{4}{c}{Import} \\
			& 1st   & 2nd   & 3rd   & 4th \\
			\midrule
			$\Delta \ln RER_{ct}$ & 0.222*** & 0.378*** & 0.404*** & 0.242*** \\
			& (0.056) & (0.042) & (0.032) & (0.020) \\
			$\Delta \ln RGDP_{ct}$ & -0.371 & 0.444* & 0.314* & -0.054 \\
			& (0.324) & (0.241) & (0.179) & (0.129) \\
			Year FE  & Yes   & Yes   & Yes   & Yes \\
			Firm-product-country FE & Yes   & Yes   & Yes   & Yes \\
			Observations & 372447 & 450728 & 492016 & 476829 \\
			\midrule
			Panel B & \multicolumn{4}{c}{Export} \\
			& 1st   & 2nd   & 3rd   & 4th \\
			\midrule
			$\Delta \ln RER_{ct}$ & 0.101*** & 0.096*** & 0.032*** & 0.007 \\
			& (0.023) & (0.014) & (0.010) & (0.008) \\
			$\Delta \ln RGDP_{ct}$ & -0.099 & 0.082 & -0.062 & -0.054 \\
			& (0.188) & (0.104) & (0.070) & (0.052) \\
			Year FE  & Yes   & Yes   & Yes   & Yes \\
			Firm-product-country FE & Yes   & Yes   & Yes   & Yes \\
			Observations & 367524 & 464827 & 508742 & 452881 \\
			\bottomrule
		\end{tabular}
		\begin{tablenotes}
			\footnotesize
			\item[*] Standard errors in parentheses; *, **, and *** indicate significance at 10\%, 5\% and 1\% levels. The dependent variable is the price change $\Delta \ln P_{ijct}$. Results for each quartile group with interaction terms of credit constraints are not provided here but are available upon request. All regressions include firm-product-country fixed effects and year fixed effects.
		\end{tablenotes}
	\end{threeparttable}
	\label{tab7.3}
\end{table}

Compared with the literature, \cite{auer2016} suggest that the direct response of prices to an exchange rate shock is U-shaped in exporter market share while \cite{devereux2017} supplement it by arguing that the market share of the importing firm is negatively correlated with pass-through and positively with local currency pricing (LCP). The insignificant relationship we find between export exchange rate pass-through and market share may be because China's original export exchange rate pass-through is nearly complete. Yet, the hump-shaped relationship between Chinese importers is interesting and worthy of further discussion.

\section{Firm Heterogeneity in Ownership} \label{sec-7.3}

An important feature of the financing environment in China is that links to the government and ease of access to credit are highly correlated. Therefore, ownership may be another potential factor affecting credit constraints, as state-owned enterprises (SOEs), domestic private firms, foreign-owned firms and joint ventures in China have different credit features in addition to their heterogeneous credit demand at the industry level. In this section, we check different types of firms and see whether their ownership types would affect the current main results. First, we provide baseline estimations of exchange rate pass-through for those firms with different ownership. Next, we check whether credit needs have effects with the same direction on exchange rate pass-through among all those firms.

\begin{table}[htbp]
	\centering
	\caption{Estimations with Importers of Different Ownership Types}
	\label{tab7.4}
	\resizebox{1.1\linewidth}{!}{
		\begin{threeparttable}
			\begin{tabular}{lllllllll}
				\toprule
				& (1)   & (2)   & (3)   & (4)   & (5)   & (6)   & (7)   & (8) \\
				\midrule
				Panel A-1 & \multicolumn{4}{c}{State-owned Enterprises} & \multicolumn{4}{c}{Domestic Private Enterprises} \\
				& Baseline & FPC   & External  & Tangibility & Baseline & FPC   & External  & Tangibility \\
				&  &   & Finance &  &  &   &  Finance &  \\
				\midrule
				$\Delta \ln RER_{ct}$ & 0.620*** & 0.462** & 0.488*** & 1.339*** & 0.146** & 0.053 & 0.100 & 0.682*** \\
				& (0.177) & (0.186) & (0.184) & (0.360) & (0.074) & (0.078) & (0.075) & (0.169) \\
				$\Delta \ln RGDP_{ct}$ & -1.675 & -1.846 & -1.893 & -1.747 & 0.195 & 0.212 & 0.189 & 0.236 \\
				& (1.257) & (1.258) & (1.259) & (1.257) & (0.525) & (0.525) & (0.525) & (0.525) \\
				$\Delta \ln RER_{ct}$*$FPC_{j}$ &       & 0.252*** &       &       &       & 0.181*** &       &  \\
				&       & (0.095) &       &       &       & (0.049) &       &  \\
				$\Delta \ln RER_{ct}$*$ExtFin_{j}$ &       &       & 0.786*** &       &       &       & 0.448*** &  \\
				&       &       & (0.299) &       &       &       & (0.145) &  \\
				$\Delta \ln RER_{ct}$*$Tang_{j}$ &       &       &       & -2.761** &       &       &       & -2.069*** \\
				&       &       &       & (1.205) &       &       &       & (0.588) \\
				Year FE  & Yes   & Yes   & Yes   & Yes   & Yes   & Yes   & Yes   & Yes \\
				Firm-product & Yes   & Yes   & Yes   & Yes   & Yes   & Yes   & Yes   & Yes \\
				-country FE &&&&&&&& \\
				Observations     & 17537 & 17537 & 17537 & 17537 & 88729 & 88729 & 88729 & 88729 \\
				\midrule
				Panel A-2 & \multicolumn{4}{c}{Multinational Enterprises} & \multicolumn{4}{c}{Joint Ventures} \\
				& Baseline & FPC   & External  & Tangibility & Baseline & FPC & External  & Tangibility \\
				&  &   & Finance &  &  &   &  Finance &  \\
				\midrule
				$\Delta \ln RER_{ct}$ & 0.316*** & 0.008 & 0.121*** & 1.342*** & 0.402*** & 0.256*** & 0.322*** & 0.919*** \\
				& (0.021) & (0.022) & (0.021) & (0.044) & (0.024) & (0.026) & (0.024) & (0.051) \\
				$\Delta \ln RGDP_{ct}$ & 0.516*** & 0.511*** & 0.504*** & 0.519*** & 0.086 & 0.167 & 0.189 & 0.107 \\
				& (0.117) & (0.117) & (0.117) & (0.117) & (0.150) & (0.150) & (0.150) & (0.150) \\
				$\Delta \ln RER_{ct}$*$FPC_{j}$ &       & 0.501*** &       &       &       & 0.232*** &       &  \\
				&       & (0.013) &       &       &       & (0.015) &       &  \\
				$\Delta \ln RER_{ct}$*$ExtFin_{j}$ &       &       & 1.625*** &       &       &       & 0.671*** &  \\
				&       &       & (0.041) &       &       &       & (0.044) &  \\
				$\Delta \ln RER_{ct}$*$Tang_{j}$ &       &       &       & -4.138*** &       &       &       & -2.102*** \\
				&       &       &       & (0.157) &       &       &       & (0.185) \\
				Year FE  & Yes   & Yes   & Yes   & Yes   & Yes   & Yes   & Yes   & Yes \\
				Firm-product & Yes   & Yes   & Yes   & Yes   & Yes   & Yes   & Yes   & Yes \\
				-country FE &&&&&&&& \\
				Observations     & 1067958 & 1067958 & 1067958 & 1067958 & 617796 & 617796 & 617796 & 617796 \\
				\bottomrule
			\end{tabular}
			\begin{tablenotes}
				\footnotesize
				\item[*] Standard errors in parentheses; *, **, and *** indicate significance at 10\%, 5\% and 1\% levels. The dependent variable is the price change $\Delta \ln P_{ijct}$. Columns (2)-(4) use different measures of credit constraints calculated using U.S. data. Panel A-1 and panel A-2 show results of import ERPT for state-owned enterprises, domestic private enterprises, multinational enterprises, and joint venture firms, respectively. Panel B-1 and panel B-2 show the results of export ERPT for those firms.  Panel B is shown in Appendix \ref{tabA.4}. All regressions include firm-product-country fixed effects and year fixed effects.
			\end{tablenotes}
		\end{threeparttable}
	}
\end{table}

Table \ref{tab7.4} shows the results for importers of different ownership types and Table \ref{tabA.4} shows the results for exporters. From the baseline coefficients in columns (1) and (5) in both panels of Table \ref{tab7.4}, we find that state-owned enterprises (SOEs) have the most complete import exchange rate pass-through around 62\%, while domestic private enterprises have the most incomplete import pass-through at about 14.6\%. Firms with foreign capital (including multinational enterprises and joint ventures) perform in the middle, with 31.6\% and 40.2\% import ERPT respectively. As for the impact of industry-level credit demand constraints on exchange rate pass-through (columns (2)-(4) and (6)-(8) in both panels), the signs and significance of regression coefficients for all types of firms are the same as in the previous main results. This suggests that although ownership type itself can cause differences among firms, the effect of credit constraints on exchange rate pass-through cannot be ignored in all types of firms.

\section{Discussion on Import-export Linkage} \label{sec-7.4}

Following the analysis of two-way traders, we briefly discuss the potential relationship between import and export pass-through of exchange rate shocks here. On the one hand, for two-way traders, export exchange rate pass-through could act as a "pressure-reducing valve" for import price pass-through. When a firm has the ability to pass more exchange rate fluctuations to destination prices, it has more room to absorb price fluctuations of imported inputs. In other words, the firm-level export pass-through will have a positive effect on all product-level import pass-throughs of the same firm.

On the other hand, big importers are also big exporters (AIK\cite{aik2014}). Therefore, advantages in some firm characteristics, either explicit ones such as size, market share, or productivity, or implicit ones like foreign networks may lead them to have greater bargaining power on both the import side and the export side and thus cause less complete export and import price pass-through at the same time. 

To study whether and how those two channels affect the import exchange rate pass-through, ideally we need to control export price pass-through when calculating import price pass-through for all two-way traders. However, it is not available to estimate the price pass-through of each individual firm. We could only check potential influential factors individually using current strategies. However, this provides a future direction for studies to determine the factors of import pass-through.

\section{Discussion on the Trend of China’s Exchange Rate Pass-through}

This article focuses on the horizontal comparison of import and export exchange rate pass-through and its causes. In the future, we can extend our methodology to reveal the time-series trend by varying periods. Actually, in our preliminary research, we found that China's export exchange rate pass-through has shown an overall downward trend (to be more incomplete) during the 2000s while the trend of China's import pass-through is vague. However, a more detailed investigation requires a longer panel covering recent years.

\cite{devereux2017} attempt to study how changes in import market shares over time may be related to changes in aggregate exchange rate pass-through over time. They run weighted rolling regressions on 12-month windows moving up by month covering 70 months. Although they find no perfect coincidence between the increase in import market share of large importers and the decrease in pass-through, the general trend does show that the aggregate pass-through is incomplete when the import market share of large importers is large. Therefore, we have reason to suspect that the evolution in China's exchange rate pass-through over time may also be due to the concentration of market share, that is, large exporters and importers occupy more market shares and may thus have stronger bargaining power.

To further ask whether the trend of exchange rate pass-through could be at least partially affected by credit constraints, there are two possible channels to discuss. First, the credit constraints on Chinese exporters are gradually loosening. It may be because of the decreasing credit needs of Chinese exporters or the improvement of the immature financial market in China. Second, China's exports switch from more credit-constrained to less-constrained industries. Credit-constrained firms find it harder to survive in export markets (extensive margin) or export less in value (intensive margin).
